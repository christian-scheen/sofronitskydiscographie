\chapter[%
Introduction][%
Introduction]{%
Introduction}
\label{chap:Introduction}

\section{Disposition et conventions du document}

Ce document s'efforce de recenser tous les enregistrements connus du
pianiste russe \VSofronitsky{} (\Dates{1901-05-08}{1961-08-29}).
Il utilise la translittération scientifique de l'alphabet cyrillique russe
pour les noms propres et noms de lieux~; un tableau de correspondance avec
la transcription française se trouve en page~\pageref{tab:Correspondance}.

L'organisation de la discographie est d'abord alphabétique sur le nom du
compositeur, ensuite alphabétique sur le titre de l'œuvre -- seuls les noms
communs sont pris en considération --, et enfin chronologique sur la date de
l'enregistrement.
Les dates sont renseignées sous la forme non ambiguë \citet{ISO8601} chaque
fois que cela est possible, \cad \Quote{AAAA-MM-JJ}~; avec cette notation,
le classement par ordre lexicographique correspond à l'ordre chronologique,
ce qui constitue un second avantage pour une discographie.

Il y a un chapitre par compositeur (numérotation principale~: 5,~etc.), une
section par œuvre (numérotation secondaire~: 5.1,~etc.) et une énumération
qui en regroupe les interprétations connues (numérotation entre crochets~:
[5.1A],~etc.)~; bien entendu, le compteur des œuvres est remis à zéro pour
chaque compositeur et celui des interprétations est remis à zéro pour chaque
œuvre.

On peut accéder à la discographie de plusieurs manières~:
\begin{itemize}
 \item
 \emph{via} la liste ou index des compositeurs et des œuvres classées par
 ordre croissant des numéros d'opus en page~\pageref{chap:Oeuvres}.
 Cette liste complète la table des matières placée en fin de document~;
 \item
 \emph{via} la liste ou index chronologique des enregistrements en studio et
 en récital en page~\pageref{chap:Chronologie}~;
 \item
 \emph{via} la table des matières en page~\pageref{chap:Tabledesmatieres} --
 qui est, pour l'essentiel, la liste alphabétique des compositeurs et des
 œuvres.
\end{itemize}

Le répertoire pianistique de \VSofronitsky{}, bien plus vaste que ce qui
nous en est parvenu sous la forme d'enregistrements, est présenté en
page~\pageref{chap:Repertoire}.
Les récitals donnés par \VSofronitsky{} lors de son séjour à Paris à la fin
des années~1920 font l'objet d'un chapitre en page~\pageref{chap:Paris}.
Une chronologie des récitals se trouve en page~\pageref{chap:Recitals}.

Un chapitre en page~\pageref{chap:Studio} est consacré aux sessions
d'enregistrements en studio.
Le contenu des disques vinyles -- série d'enregistrements complets, publiée
par Melodija --, des disques compacts et des disques 78~tours est renseigné
et répertorié, en page~\pageref{chap:Contenu}, à la fin de la partie de
discographie et sous la forme d'un index.

Le travail d'une discographie n'est jamais terminé, et jamais tout à fait
correct non plus~: un état des lieux en page~\pageref{chap:Etat-des-lieux}
dresse l'inventaire des disques compacts non répertoriés et des
enregistrements indisponibles ou non publiés.
Une bibliographie en page~\pageref{chap:Bibliographie}, une liste ou index
des citations en page~\pageref{chap:Listedescitations}, un index des noms
propres cités en page~\pageref{chap:Indexonomastique} et un lexique en
page~\pageref{chap:Lexique} complètent le document.
Le colophon en page~\pageref{chap:Colophon} indique les modalités d'écoutes
comparatives, la chaîne de fabrication typographique et les polices de
caractères utilisées dans le document.

Les textes en \textcolor{dvsred}{rouge} (liens internes normaux), en
\textcolor{dvsgreen}{vert} (citations bibliographiques dans le texte) ou en
\textcolor{dvsblue}{bleu} (URL~qui ouvrent des fichiers locaux ou non) sont
des hyper\-liens qui permettent de se déplacer dans le document ou à partir
de celui-ci.
Les disques 78~tours sont représentés par le symbole~\symbSE, les disques
vinyles par~\symbLP, les disques compacts par~\symbCD et les éditions
dématérialisées par~\symbDO (fichiers~\href{https://xiph.org/flac/}{FLAC}
ou~\href{http://mpeg.chiariglione.org/standards/mpeg-2/audio}{MP3}, par
exemple).

\section{Références discographiques}

Il existe plusieurs références à propos de l'héritage discographique de
\VSofronitsky{}.
On peut néanmoins relever les cinq sources les plus importantes~:
\begin{itemize}
 \item
 la référence principale est la discographie remarquable établie par
 \FMalik{} \citep[voir][]{Malik} et publiée dans le numéro de l'automne~1998
 de la revue anglaise \emph{International Piano Quarterly}~(IPQ).
 Le présent document doit beaucoup à la discographie de \FMalik{} et
 n'existerait pas sans elle~: il reprend toutes les données fournies par
 \FMalik{} et s'efforce de les compléter en fonction des éditions
 postérieures à~1998, année de parution de la discographie dans
 \emph{International Piano Quarterly}~;
 \item
 la deuxième référence importante est la thèse de doctorat~(DMA) consacrée
 par \EWhite{} au pianiste \citep[voir][]{White}~; elle contient en
 particulier une chronologie détaillée et les programmes de récitals
 \citep[voir][p.~40-71]{White}, la liste du répertoire du pianiste
 \citep[voir][p.~72-82]{White} et enfin une discographie historique des
 publications qui ont été effectuées sur disque vinyle
 \citep[voir][p.~83-107]{White}.
 Cette thèse est disponible auprès des \emph{International Piano Archives}
 de l'université du Maryland (IPAM, College Park).
 Le présent document doit aussi beaucoup à ce travail de doctorat pionnier
 en Occident~;
 \item
 la troisième référence fondamentale est la discographie établie en russe
 par \INikonovich{} \citep[voir][]{Nikonovich11}~; elle concerne aussi bien
 les éditions sur disque vinyle \citep[voir][\hbox{p.~1-11}]{Nikonovich11}
 que les publications sur disque compact
 \citep[voir][p.~11-30]{Nikonovich11}.
 Cette discographie a été reprise par \citet[p.~375-392]{Scriabine}.
 Le travail d'\INikonovich{} a permis de vérifier, de corriger et surtout de
 compléter de manière très importante les données du présent document, en
 particulier en ce qui concerne les disques vinyles \Quote{isolés}, la série
 de disques compacts consacrés par Vista Vera aux enregistrements au musée
 \Scriabine{} et les disques 78~tours~;
 \item
 la quatrième référence importante est le vaste document compilé par
 \ARossi{} \citep[voir][]{Rossi}, établi en partie sur la base de la
 discographie due à \INikonovich{} \citep[voir][]{Nikonovich11}~; on y
 trouve en particulier la traduction, en anglais ou en italien, des notes
 d'\INikonovich{} rédigées en russe, suivie d'une section assez détaillée
 concernant les éditions sur disque compact~;
 \item
 la cinquième référence essentielle est la discographie de \VSofronitsky{}
 établie par ordre chronologique sur la date de l'enregistrement, et
 maintenue à jour par \CJohansson{} \citep[voir][]{Johansson}~; cette
 discographie concerne à la fois les enregistrements commerciaux du pianiste
 et les bandes non commerciales telles, par exemple, les captations qui ont
 été effectuées au musée \Scriabine{} à Moskva.
 On y trouve, en particulier, les références de plusieurs enregistrements
 privés qui ne sont mentionnés dans aucune autre source et des précisions à
 propos de plusieurs dates d'enregistrement qui étaient mal connues voire
 inconnues jusque là.
\end{itemize}

Il existe aussi des références complètes à propos des séries éditées par les
firmes Melodija \citep[voir][]{Malik, Masuda, Nikonovich11, White}, série
inachevée sur disque vinyle de~1979 à~1984, Denon \citep[voir][]{Denon03,
Denon05, Denon06}, sur disque compact de~1996 à~2000, avec des rééditions
partielles récentes en mai~2003 \citep{Denon03}, mai~2005 \citep{Denon05} et
novembre~2006 \citep{Denon06}, et Vista Vera \citep[voir][]{VistaVera}, sur
disque compact à partir de~1997, pour la série principale, et de~2009, pour
la série des enregistrements effectués au musée \Scriabine{}.
\HFogel{} \citep[voir][]{Fogel}, \SGraham{} \citep[voir][]{Graham} et
\PTaylor{} \citep[voir][]{Taylor} fournissent des informations importantes à
propos de diverses rééditions, surtout sur disque compact.
Sur la toile, un certain nombre de sites proposent des bases de données plus
ou moins fiables pour les rééditions d'enregistrements sur disque compact
\citep[voir][]{Discogs.com, Freedb.org, Gracenote.com}.
Les listes de vente, et en particulier de vente aux enchères, fournissent
parfois des informations complètes à propos de leurs objets.
Les sites des médiathèques publiques proposent en général un moteur de
recherche pour accéder au contenu et parfois à des extraits des disques de
leurs collections.
Indiquons que de nombreux disques compacts sont constitués, en tout ou en
partie, d'enregistrements au préalable inédits voire inconnus dans la
discographie du pianiste (Arbiter ARB~157~; Classical Records CR-014~; Denon
COCO-80188~; Denon COCO-80189~; Denon COCO-80385/6~= COCQ-83669/70~; Denon
COCO-80568~= COCQ-83671~; Denon COCQ-83286~= COCQ-83971~; Melodija GP15926~;
Moscow State Conservatoire SMC CD~0019~; Moscow State Conservatoire SMC
CD~0020~; Prometheus Editions EDITION003~; Russian Disc RD CD~15001~; Vista
Vera VVCD-97014~= VVCD-00014~; Vista Vera VVCD-00093~; Vista Vera
VVCD-00148-2~; Vista Vera VVCD-00195~; Vista Vera VVCD-00203~; Vista Vera
VVCD-00204~; Vista Vera VVCD-00210~; Vista Vera VVCD-00213~; Vista Vera
VVCD-00218~; Vista Vera VVCD-00222~; Vista Vera VVCD-00223~; Vista Vera
VVCD-00224~; Vista Vera VVCD-00225~; Vista Vera VVCD-00233~; Vista Vera
VVCD-00241~; Vista Vera VVCD-00248~; Vista Vera VVCD-00249).

\section{Orientations bibliographiques}

\subsection{Ouvrage coordonné par \citeauthor{Milshteyn82a}}

Après le décès d'un personnage célèbre, il est de tradition, en Rossija, de
lui rendre hommage en publiant un recueil de souvenirs ou de réminiscences
\citep[voir][p.~xi]{White}.
L'ouvrage de ce type consacré à \VSofronitsky{}, placé sous la direction de
\citeauthor{Milshteyn82a} -- première édition en~\citeyear{Milshteyn70} et
seconde édition en~\citeyear{Milshteyn82a} --, regroupe les souvenirs de
\citet{Milshteyn82b, Sofronitsky82a, Sofronitsky82b, NeuhausH82, Argamakov,
Bogdanov82, Golubovskaya, Savshinsky82, Yudina82, Nekrasova82,
Miklashevskaya, Oborine82, Zak82, Geronimus, Shershevsky, Modyel82,
Shaborkina, Delson82, Tolstoi, Nikonovich82, NeuhausS82, Bashkirov82,
Berman82, Sturtsel, Podolskaya, Lobanov82, Zhukova82, Moroshkina, Bragina,
Alekseiev82, Gorohovsky, Savkevich, Mozhanskaya, Adzhemov, Smirnov,
Panarine, Rumyantsev, Shiryaeva}.
Sept de ces contributions sont en outre disponibles sur Internet
\citep[voir][]{Bashkirov10, Berman10, Neuhaus10, Oborine10, Savshinsky10,
Yudina10, Zak10}~; celle de \MYudina{} a en outre été traduite en anglais
\citep[voir][]{Yudina02}.

Voir tableau~\ref{tab:Milshteyn} pour la liste des premières publications
de ces contributions.

\begin{table}[!htbp]
 \fontsize{10}{12.6pt}\selectfont
 \centering
 \caption{\citet{Milshteyn82a}~: liste des premières publications}
 \label{tab:Milshteyn}
 \begin{tabular}{ll}
  \toprule
  \textbf{Contribution} & \textbf{Première publication} \\
  \midrule
  \citet{Sofronitsky82a} & Rédigée fin des années~1930, archive familiale \\
  \citet{Sofronitsky82b} & Rédigée en~1964-1965, contribution originale \\
  \citet{NeuhausH82} & \emph{Советская культура}, 27~mai~1961 \\
  \citet{Argamakov} & Rédigée en~1962, contribution originale \\
  \citet{Bogdanov82} & \citet{Bogdanov65, Bogdanov67a} \\
  \citet{Golubovskaya} & Rédigée en~1962, contribution originale \\
  \citet{Savshinsky82} & \citet{Savshinsky61} (ajouts et modifications) \\
  \citet{Yudina82} & Rédigée en~1965, contribution originale \\
  \citet{Nekrasova82} & Contribution originale pour la~2\ieme{} édition \\
  \citet{Miklashevskaya} & Rédigée en~1964, contribution originale \\
  \citet{Oborine82} & Rédigée en~1964, contribution originale \\
  \citet{Zak82} & Rédigée en~1965, contribution originale \\
  \citet{Geronimus} & Rédigée en~1965, contribution originale \\
  \citet{Shershevsky} & Rédigée en~1972 pour la~2\ieme{} édition \\
  \citet{Modyel82} & Rédigée en~1979 pour la~2\ieme{} édition \\
  \citet{Shaborkina} & Rédigée en~1962, contribution originale \\
  \citet{Delson82} & Rédigée en~1963-1964, contribution originale \\
  \citet{Tolstoi} & Rédigée en~1963, contribution originale \\
  \citet{Nikonovich82} & Rédigée dans les années~1960, étude originale \\
  \Ibid & \citet{Nikonovich61, Nikonovich68a, Nikonovich68b} (fragments) \\
  \citet{NeuhausS82} & Pour le~70\ieme{} anniversaire de naissance de~VVS \\
  \Ibid & \citet{Neuhaus88} (rééd.), \citet{White} (trad.) \\
  \citet{Bashkirov82} & Rédigée en~1977 pour la~2\ieme{} édition \\
  \citet{Berman82} & Contribution originale pour la~1\iere{} édition \\
  \citet{Sturtsel} & Rédigée en~1964, contribution originale \\
  \citet{Podolskaya} & Rédigée en~1963, contribution originale \\
  \citet{Lobanov82} & Rédigée en~1963-1964, contribution originale \\
  \citet{Zhukova82} & Rédigée en~1964, contribution originale \\
  \citet{Moroshkina} & Rédigée en~1964, contribution originale \\
  \citet{Bragina} & Rédigée en~1962, contribution originale \\
  \citet{Alekseiev82} & Rédigée dans les années~1960, étude originale \\
  \Ibid & Reprise dans les années~1970 pour la~2\ieme{} édition \\
  \citet{Gorohovsky} & Rédigée en~1965, contribution originale \\
  \citet{Savkevich} & Rédigée en~1962, contribution originale \\
  \citet{Mozhanskaya} & Rédigée dans les années~1960, étude originale \\
  \citet{Adzhemov} & Contribution originale \citep[voir][]{Adzhemov61} \\
  \citet{Smirnov} & Rédigée en~1977 pour la~2\ieme{} édition \\
  \citet{Panarine} & Rédigée en~1964, contribution originale \\
  \citet{Rumyantsev} & Rédigée en~1963, contribution originale \\
  \citet{Shiryaeva} & Rédigée en~1963, contribution originale \\
  \bottomrule
 \end{tabular}
\end{table}

\subsection{Ouvrage coordonné par \citeauthor{Nikonovich08}}

L'ouvrage en russe dirigé par \citet{Nikonovich08} regroupe de nombreuses
contributions passionnantes~; l'article en deux parties de \VVoskobojnikov{}
présente cet ouvrage de manière détaillée et en propose de larges extraits
traduits en italien \citep[voir][]{Voskobojnikov09a, Voskobojnikov09b}~:
\begin{itemize}
 \item
 un avant-propos par \ASScriabine{} \citep{Scriabine08}~;
 \item
 les souvenirs d'\ISofronitskaya{}, la veuve du fils du pianiste
 \citep{Sofronitskaya08}.
 Contribution rédigée spécialement pour cette édition~;
 \item
 les souvenirs de \RKoganSofronitskaya{}, la fille aînée du pianiste, avec
 en particulier des extraits des lettres envoyées par ce dernier à sa
 première épouse \citep{Kogan08}.
 Contribution rédigée spécialement pour cette édition~;
 \item
 le vaste essai d'\INikonovich{}, ami intime et élève du maître durant les
 cinq dernières années de la vie de celui-ci \citep{Nikonovich08a}.
 Version étendue d'un texte qui avait été publié pour la première fois dans
 l'ouvrage collectif dirigé par \citet[p.~223-340]{Milshteyn70}~;
 \item
 le long article de \VNekrasova{}, consacré au parcours artistique du
 pianiste et reprenant une chronologie de ses programmes de récitals
 \citep{Nekrasova08}.
 Document publié pour la première fois dans l'ouvrage en question~;
 \item
 la contribution du pianiste \VMerzhanov{} \citep{Merzhanov08}.
 Rédigée spécialement pour cette édition~;
 \item
 la contribution du pianiste \LNaumov{} \citep{Naumov08}.
 Publiée pour la première fois en~2002, dans un ouvrage intitulé \emph{Sous
 le signe de \Neuhaus{}} \citep[p.~73-82]{Naumov02}~;
 \item
 la contribution de \SBenditsky{}, élève de \HNeuhaus{} \citep{Benditsky08}.
 Publiée dans la revue \emph{\foreignlanguage{russian}{Волга}}, 1995,
 \Number{7-9}~;
 \item
 l'article de \VGornostaieva{}, professeur au conservatoire de Moskva
 \citep{Gornostaieva08}.
 Rédigé pour le centième anniversaire de la naissance de \VSofronitsky{} et
 publié dans la revue \emph{\foreignlanguage{russian}{Музыкальная жизнь}},
 2001, \Number{4} \citep{Gornostaieva01}~;
 \item
 la contribution de \LSosina{}, collègue de \VGornostaieva{}
 \citep{Sosina08}.
 Rédigée spécialement pour cette édition~;
 \item
 le témoignage d'\OBoshniakovich{}, pianiste d'origine serbe, élève de
 \HNeuhaus{} et de \KIgumnov{} \citep{Boshniakovich08}.
 Publié dans la revue \emph{\foreignlanguage{russian}{Музыка и время}},
 2005, \Number{11}, p.~16-18~;
 \item
 les souvenirs d'\OZhukova{}, élève du maître entre~1942 et~1950
 \citep{Zhukova08}.
 Publiés pour la première fois dans l'ouvrage collectif dirigé par
 \citet[p.~406-430]{Milshteyn70}~;
 \item
 les souvenirs de \TBernblium{}, élève du maître \citep{Bernblium08}.
 Rédigés spécialement pour cette édition~;
 \item
 les souvenirs de \JGrigorian{}, élève du maître à l'\hbox{école} centrale
 de musique et ensuite au conservatoire \citep{Grigorian08}.
 Rédigés en~2001 spécialement pour cette édition et publiés avec des
 modifications éditoriales mineures~;
 \item
 les souvenirs de \NKalinenko{}, née \Feigina{}, élève du maître que l'on
 peut écouter sur le disque compact édité par Prometheus Editions EDITION003
 (plages~1 à~20) \citep{Kalinenko08}.
 Rédigés en~2000 spécialement pour cette édition et publiés avec des
 modifications éditoriales mineures~;
 \item
 les souvenirs de \NKorsakova{}, élève du maître dès~1951
 \citep{Korsakova08}.
 Rédigés en~2001 spécialement pour cette édition et publiés avec des
 modifications éditoriales mineures~;
 \item
 les souvenirs de \VKochkina{}, élève du maître dès~1944 \citep{Kochkina08}.
 Rédigés en~2001 spécialement pour cette édition et publiés avec des
 modifications éditoriales mineures~;
 \item
 l'article de \PLobanov{}, musicien, technicien de restauration des bandes
 audio, auteur des enregistrements des leçons de \HNeuhaus{} et
 \VSofronitsky{}, collaborateur du musée \Scriabine{} et élève du maître dès
 novembre~1942 \citep{Lobanov08a}.
 L'article offre en particulier un exposé détaillé du Scherzo en \kB mineur
 de \Chopin{} et de la Novelette en \kE majeur de \Schumann{} avec les
 indications de \VSofronitsky{}.
 Publié pour la première fois dans l'ouvrage collectif dirigé par
 \citet[p.~386-405]{Milshteyn70}~; publié avec les modifications et ajouts
 apportés par l'auteur, en~2000, spécialement pour cette édition~;
 \item
 les souvenirs de \DNudelman{} \citep{Nudelman08}.
 Rédigés en~2001 spécialement pour cette édition et publiés avec des
 modifications éditoriales mineures~;
 \item
 les souvenirs d'\OTrazevskaya{}, élève du maître dès~1944
 \citep{Trazevskaya08}.
 Rédigés en~2001 spécialement pour cette édition et publiés en manuscrit
 avec des modifications éditoriales mineures~;
 \item
 les souvenirs de \LFichtengoltz{}, élève de \HNeuhaus{} et qui fréquenta
 pendant un an, en~1943, la classe de \VSofronitsky{}
 \citep{Fichtengoltz08}.
 Rédigés en~2001 spécialement pour cette édition et publiés avec des
 modifications éditoriales mineures~;
 \item
 le long article de \DRabinovitch{}, musicologue et critique musical
 \citep{Rabinovich08}.
 Publié pour la première fois en~1962, dans un ouvrage intitulé
 \emph{Portraits de pianistes} \citep[voir][]{Rabinovich62}~;
 \item
 le long article de \LGakkel{}, musicologue et critique musical
 \citep{Gakkel08}.
 Publié pour la première fois en~1995, dans un ouvrage intitulé \emph{La
 Grandeur de l'art de la scène~: M.V.\@~\Yudina{} et V.V.\@~\Sofronitsky{}}
 \citep[voir][]{Gakkel95}~;
 \item
 l'intervention de \VChinaiev{}, pianiste et musicologue \citep{Chinaiev08}.
 Publiée d'après le texte d'un discours prononcé lors de la conférence
 consacrée au centième anniversaire de la naissance de \VSofronitsky{}~;
 \item
 l'article d'\AZolotov{}, journaliste, scénariste et critique musical
 \citep{Zolotov08}.
 Publié d'après le texte d'un discours prononcé lors de la conférence
 consacrée au centième anniversaire de la naissance de \VSofronitsky{}~;
 \item
 la contribution du pianiste \MKonchalovsky{} \citep{Konchalovsky08}.
 Publiée pour la première fois en~2002, dans un ouvrage intitulé
 \emph{\foreignlanguage{russian}{Созвучие}} [\emph{Consonance}] et publié à
 Dubna aux Éditions Feniks+~;
 \item
 l'article de \RYourenev{}, cinéaste et critique \citep{Yurenev08}.
 Rédigé spécialement pour cette édition~;
 \item
 l'article de \YBorisov{} \citep{Borisov08}.
 Fragment d'un ouvrage intitulé \emph{\foreignlanguage{russian}{По
 направлению к Рихтеру}} [\emph{Du Côté de chez \Richter{}}], publié en~2000
 à Moskva~;
 \item
 le second article d'\INikonovich{}, consacré aux enregistrements de
 \VSofronitsky{} \citep{Nikonovich08b}.
 Publié pour la première fois en~1980 sous la forme d'annotation pour la
 collection complète des enregistrements de \Sofronitsky{} chez Melodija~;
 \item
 le second article de \PLobanov{}, consacré aux rapports que le pianiste a
 entretenus avec les techniques d'enregistrement \citep{Lobanov08b}.
 Rédigé spécialement pour cette édition~;
 \item
 l'article de \TBadeyan{}, consacré à la restauration des enregistrements du
 pianiste qui ont été effectués au musée \Scriabine{} \citep{Badeyan08}.
 Rédigé spécialement pour cette édition~;
 \item
 les extraits du \emph{Journal} de \SProkofiev{} qui concernent les
 rencontres du compositeur avec le pianiste à Paris en~1928 et en~1929
 \citep{Prokofiev08}.
 Le \emph{Journal} couvre la période de~1907 à~1933 et a été publié
 intégralement en russe pour la première fois en France en~2002 par le fils
 aîné du compositeur~;
 \item
 les documents qui concernent l'amitié entre le pianiste et le metteur en
 scène \VMeyerhold{}, fusillé en~1940 \citep{NikonovichScriabine08}.
\end{itemize}

Voir tableau~\ref{tab:Nikonovich} pour la liste des premières publications
de ces contributions.

\begin{table}[!htbp]
 \fontsize{10}{12.2pt}\selectfont
 \centering
 \caption{\citet{Nikonovich08}~: liste des premières publications}
 \label{tab:Nikonovich}
 \begin{tabular}{ll}
  \toprule
  \textbf{Contribution} & \textbf{Première publication} \\
  \midrule
  \citet{Sofronitskaya08} & Contribution originale \\
  \citet{Kogan08} & Contribution originale \\
  \citet{Nikonovich08a} & \citet[p.~223-340]{Milshteyn70} et ajouts \\
  \citet{Nekrasova08} & Contribution originale \\
  \citet{Merzhanov08} & Contribution originale \\
  \citet{Naumov08} & \citet[p.~73-82]{Naumov02} \\
  \citet{Benditsky08} & In~: \emph{Волга}, 1995, \Number{7-9} \\
  \citet{Gornostaieva08} & \citet{Gornostaieva01} \\
  \citet{Sosina08} & Contribution originale \\
  \citet{Boshniakovich08} & \citet{Boshniakovich05} \\
  \citet{Zhukova08} & \citet[p.~406-430]{Milshteyn70} \\
  \citet{Bernblium08} & Contribution originale \\
  \citet{Grigorian08} & Contribution originale (rédigée en~2001) \\
  \citet{Kalinenko08} & Contribution originale (rédigée en~2000) \\
  \citet{Korsakova08} & Contribution originale (rédigée en~2001) \\
  \citet{Kochkina08} & Contribution originale (rédigée en~2001) \\
  \citet{Lobanov08a} & \citet[p.~386-405]{Milshteyn70} et ajouts en~2000 \\
  \citet{Nudelman08} & Contribution originale (rédigée en~2001) \\
  \citet{Trazevskaya08} & Contribution originale (rédigée en~2001) \\
  \citet{Fichtengoltz08} & Contribution originale (rédigée en~2001) \\
  \citet{Rabinovich08} & \citet{Rabinovich62} \\
  \citet{Gakkel08} & \citet{Gakkel95} \\
  \citet{Chinaiev08} & Conférence pour le~100\ieme{} anniversaire de~VVS \\
  \citet{Zolotov08} & Conférence pour le~100\ieme{} anniversaire de~VVS \\
  \citet{Konchalovsky08} & \citet{Konchalovsky02} \\
  \citet{Yurenev08} & Contribution originale \\
  \citet{Borisov08} & \citet{Borisov00} \\
  \citet{Nikonovich08b} & \citet{Nikonovich79} \\
  \citet{Lobanov08b} & Contribution originale \\
  \citet{Badeyan08} & Contribution originale \\
  \citet{Prokofiev08} & \citet{Prokofiev02} \\
  \bottomrule
 \end{tabular}
\end{table}

\subsection{Ouvrage coordonné par \citeauthor{Scriabine}}

L'ouvrage en russe dirigé par \citet{Scriabine} est scindé en sept parties,
précédées d'une introduction par \citeauthor{ScriabineNikolaieva}.
La première s'attache à la signification et à la portée de l'art du pianiste
et regroupe les sept contributions de \citeauthor{Vedernikov13},
\citeauthor{Smirnov13}, \citeauthor{Orlovsky13}, \citeauthor{Spiridonova13},
\citeauthor{Sokolov13}, \citeauthor{Kuznetsov13} et \citeauthor{Yudina13}.
La deuxième partie se consacre à son style d'exécution, en particulier dans
la musique de \Scriabine{} et de \Schumann{}, et comprend les articles de
\citeauthor{Nikolaiev13}, \citeauthor{Alekseiev13a}, \citeauthor{Zinger13}
et \citeauthor{Merkoulov13}.
La troisième partie évoque son travail de pédagogue, avec les contributions
de \citeauthor{Scriabine13a}, \citeauthor{Zhukova13},
\citeauthor{Salnikov13} et \citeauthor{Eshpai13}.
La quatrième partie est consacrée aux souvenirs d'une vingtaine d'auteurs~:
\citeauthor{Sofronitsky13a}, \citeauthor{Sofronitsky13b},
\citeauthor{Sofronitskaya13}, \citeauthor{Kogan13},
\citeauthor{Berkovskaya13}, \citeauthor{Askoldova13},
\citeauthor{Barinova13}, \citeauthor{Pasternak13},
\citeauthor{Alekseiev13b}, \citeauthor{Gornostaieva13},
\citeauthor{Tolstoi13}, \citeauthor{Paperno13}, \citeauthor{Kondratiev13},
\citeauthor{Muravliov13}, \citeauthor{Cherkasov13}, \citeauthor{Zhukov13},
\citeauthor{Ivanov13}, \citeauthor{Pokrovsky13}, \citeauthor{Savostiouk13}
et \citeauthor{Safonov13}.
La cinquième partie, préparée et présentée par \citeauthor{Scriabine13b},
propose septante-six lettres choisies dans la correspondance du pianiste et
de sa famille (p.~294-346).
La sixième partie est constituée d'extraits des archives de la famille
d'\EVizel{} et regroupe les articles de \citeauthor{Vizel13a},
\citeauthor{Vizel13b} et \citeauthor{Vizel13c}~; une partie importante des
illustrations de l'ouvrage provient d'ailleurs des œuvres de la famille
\Vizel{}.
Enfin, on trouve, dans la dernière partie, une discographie établie par
\INikonovich{} en pages~375-392 -- voir aussi \citet{Nikonovich11} --, une
chronologie très approfondie de la vie et de l'œuvre de \VSofronitsky{} en
pages~393-447, sans aucun doute l'énumération de ce type la plus détaillée,
et quelques autres recensements en pages~447-452.

Voir tableau~\ref{tab:Scriabine} pour la liste des premières publications
de ces contributions.

\begin{table}[!htbp]
 \fontsize{10}{13.3pt}\selectfont
 \centering
 \caption{\citet{Scriabine}~: liste des premières publications}
 \label{tab:Scriabine}
 \begin{tabular}{ll}
  \toprule
  \textbf{Contribution} & \textbf{Première publication} \\
  \midrule
  \citet{Vedernikov13} & \citet{Vedernikov01} \\
  \citet{Smirnov13} & Archive personnelle d'A.S.~Skrjabin, mai~1977 \\
  \citet{Orlovsky13} & Contribution originale~(2001), \citet{Orlovsky12} \\
  \citet{Spiridonova13} & In~: \emph{Казань}, n°~11 (2001), p.~60-63 \\
  \citet{Sokolov13} & Contribution originale~(2001), \citet{Sokolov01} \\
  \citet{Kuznetsov13} & Rédigée en~2001, contribution originale \\
  \citet{Yudina13} & \citet[p.~127-134]{Milshteyn70} \\
  \citet{Nikolaiev13} & In~: \emph{Советское искусство} (7~mai~1938) \\
  \citet{Alekseiev13a} & Odésa, 1993 \\
  \citet{Zinger13} & Rédigée en~1970, première publication \\
  \citet{Merkoulov13} & \citet[p.~302-318]{Smirnov88} \\
  \citet{Scriabine13a} & Introduction rédigée pour cette édition \\
  \citet{Zhukova13} & \citet[p.~187-214]{Sokolov68} \\
  \citet{Salnikov13} & Rédigée en~2001, contribution originale \\
  \citet{Eshpai13} & \citet{Eshpai00} \\
  \citet{Sofronitsky13a} & \citet[p.~55-86]{Milshteyn70} \\
  \citet{Sofronitsky13b} & Inconnue \\
  \citet{Sofronitskaya13} & In~: \emph{Казань}, n°~11 (2001), p.~64-67 \\
  \citet{Kogan13} & In~: \emph{Казань} (2001) \\
  \citet{Berkovskaya13} & \citet{Berkovskaya08} \\
  \citet{Askoldova13} & Contribution originale \\
  \citet{Barinova13} & In~: \emph{Музыкальная жизнь}, n°~2 (1991), p.~13 \\
  \citet{Pasternak13} & Contribution originale \\
  \citet{Alekseiev13b} & \citet[p.~449-453]{Milshteyn70}, texte complété \\
  \citet{Gornostaieva13} & \citet[p.~108-110]{Gornostaieva95} \\
  \citet{Tolstoi13} & \citet{Tolstoi95} \\
  \citet{Paperno13} & \citet[p.~69-70]{Paperno89b} \\
  \citet{Kondratiev13} & Contribution originale \\
  \citet{Muravliov13} & Contribution originale \\
  \citet{Cherkasov13} & Contribution originale \\
  \citet{Zhukov13} & \citet{Zhukov01} \\
  \citet{Ivanov13} & Contribution originale \\
  \citet{Pokrovsky13} & In~: \emph{Культура} (8~mai~2003) \\
  \citet{Savostiouk13} & Contribution originale \\
  \citet{Safonov13} & Contribution originale \\
  \citet{Vizel13a} & Contribution originale \\
  \citet{Vizel13b} & Rédigée en~2010, contribution originale \\
  \citet{Vizel13c} & Pour le~110\ieme{} anniversaire de naissance de~VVS \\
  \bottomrule
 \end{tabular}
\end{table}

\subsection{Ouvrage coordonné par \citeauthor{Lobanov03}}

L'ouvrage en russe dirigé par \citet{Lobanov03} regroupe les contributions
de dix-huit auteurs~: \citeauthor{Zolotov03a}, \citeauthor{Sofronitsky03},
\citeauthor{Lobanov03b}, \citeauthor{Sofronitskaya03},
\citeauthor{Lebedeva03}, \citeauthor{Fedorovich03}, \citeauthor{Orlovsky03},
\citeauthor{Khentova03}, \citeauthor{Tsypine03}, \citeauthor{Nikonovich03},
\citeauthor{Badeyan03b}, \citeauthor{Petropavlov03}, \citeauthor{Vizel03},
\citeauthor{Lobacheva03}, \citeauthor{Zelov03}, \citeauthor{Tompakova03a},
\citeauthor{Tompakova03b} et \citeauthor{Zolotov03b}.

\subsection{Autres publications}

\subsubsection{Travaux lexicographiques}

Plusieurs travaux lexicographiques proposent une entrée consacrée à certains
détails biographiques du pianiste et aspects de son art~: \citet{Baker,
Bellamy, Fanning01, Gurlitt, Gut, Lago, Lompech12, Milshteyn81, Paris,
Roesner, Schonberg63, Seidle, Vignal, Villemin, Wilson, Zemtsovsky,
Zilberquit}.

\subsubsection{Travaux scientifiques}

Quelques publications scientifiques -- articles, thèses et ouvrages -- sont
consacrées au pianiste ou utilisent une partie de son travail d'interprète~:
\begin{itemize}
 \item\emph{articles de revues et d'actes de conférences}.
 \citet{Brilliantova20, Girardi, Komarovskikh12a, Komarovskikh12b,
 Komarovskikh13a, Komarovskikh13b, Komarovskikh14a, Komarovskikh14b,
 Komarovskikh14c, Ljalina06, Lobanov10, Lourenco07, Lourenco10, Manyakine,
 Marquez15, Merkoulov16, Mustafaieva, Purwins, Zimogljad18}~;
 \item\emph{thèses de doctorat et de maîtrise}.
 \citet{Artese, Barolsky, Chiang, Chiao, Chkourak10, Combrink92, Cotta,
 Dejos, Komarovskikh18, Kounadi, Kurmankulov, Lajko, Lourenco05, Magalotti,
 Mehmetli, Navickaite, Ohriner, Orlovsky91, Rego, Sukhina, White,
 Zaborowski}~;
 \item\emph{ouvrages et monographies de recherche}.
 \citet{Alekseiev93, Delson70, Gakkel95, Orlovsky14, Rabinovich70,
 Rybakova06}.
\end{itemize}

\subsubsection{Monographies générales}

Certaines monographies évoquent divers aspects de la vie et de l'art du
pianiste~: \citet{Alekseiev88, Ballard, Berman03, Chernikov, Ciammarughi,
Dubal84, Dubal05, Kehler, Khentova64, Lapin, Leikin, Letourneur18,
Monsaingeon, Paperno03, Rabinovich79, Rattalino83, Rattalino90, Scriabina,
Silverman, Volkov79, Volkov97}.

\subsubsection{Littérature en russe}

De nombreuses publications en russe sont consacrées à \VSofronitsky{}~:
\begin{itemize}
 \item\emph{années d'étude et de formation}.
 \citet{Alshvang39, Bogdanov67b, Milshteyn65, Modyel88, Strelnikov}~;
 \item\emph{analyses du jeu et de la technique pianistique}.
 \citet{Afanassiev01, Blagoj00, Cheblokova, Chigareva10, Chinaiev95,
 Chinaiev17, Chitruk97, Chitruk10, Chopin10, Dubinina96, Kiseleva99,
 Lukjanova05, Neuhaus88, Orlovsky07, Sholpo96, Tropp01, Tsypine06,
 Zhukova03}~;
 \item\emph{analyses des récitals et des enregistrements}.
 \citet{Aleksandrova52, Badeyan02, Cereteli04, Drozdov35, Drozdov46,
 Gakkel86, Karatygin, Kazanskaja95, Lobanov02b, Lobanov07, Nestieva78,
 Nikitine68, Shliefstein45, Shliefstein, Strelnikov23, Zitomirsky46}.
 On peut aussi citer \citet[p.~371-388]{Nikonovich08}, pour la période
 de~1922 à~1938~;
 \item\emph{analyses du travail pédagogique}.
 \citet{Cheblokova, Fedorovich14, Orlovsky14b, Sturtsel03, Zhukova68}~;
 \item\emph{souvenirs relatifs au pianiste}.
 \citet{Bajaxunova14, Golubovskaya94, Lobanov98, Perelman99, Richter87,
 Richter00, Scriabine19, Sofronitsky00, Trazevskaya09, Vasilieva08,
 Vizel96}~;
 \item\emph{activité épistolaire du pianiste}.
 \citet{Krasilchtchik80, Kuznetsov95, Nekrasova94, Nekrasova95,
 Ramazanova97, Sofronitsky01}.
 Les sources principales demeurent cependant \citet[p.~13-34]{Kogan08}
 (période~1934-1941), \citet[p.~190-197]{Nekrasova08} (période~1936-1949),
 \citet[p.~360-367]{NikonovichScriabine08} (avec \VMeyerhold{}) et
 \citet[p.~289-346]{Scriabine13b} (période~1927-1956, correspondance
 familiale)~;
 \item\emph{concerts, festivals et concours organisés en mémoire du
 pianiste}.
 \citet{Birjukov01, Petropavlov01, Shelepova01, Zelenev01, Zhukova02}.
\end{itemize}

\subsubsection{Notes rédigées par \citeauthor{Nikonovich79} pour Melodija}
\label{sssec:NotesNikonovich}

Les notes de la série des enregistrements complets du pianiste, publiée par
l'éditeur Melodija, ont été rédigées par \citet{Nikonovich79}~; elles ont
été traduites en anglais par \citet[p.~\hbox{2-22}]{White}.
On les retrouve, en tout ou en partie, dans les livrets qui accompagnent
certains disques compacts, en particulier chez les éditeurs Classound en
russe, Melodija en russe, en anglais et en français, et Arbiter en anglais.

\subsubsection{Articles de périodiques}

De nombreux articles ont été consacrés à \VSofronitsky{}, en particulier
dans la revue \foreignlanguage{russian}{Советская Музыка} (Sovetskaja
Muzyka~: en français, \emph{Musique soviétique}), renommée en~1992
\foreignlanguage{russian}{Музыкальная академия} (Muzykal'naja akademija~:
en français, \emph{Académie de musique})~:
\begin{itemize}
 \item\emph{\foreignlanguage{russian}{Советская Музыка} et
 \foreignlanguage{russian}{Музыкальная академия}}.
 \citet{Adzhemov59, Adzhemov60, Adzhemov61, Barenboim47, Bashkirov61,
 Bogdanov65, Delson34, Gakkel72, Milshteyn51, Nikolaiev49, Nikonovich61,
 Nikonovich68a, Nikonovich68b, Nikonovich69, Nikonovich77, Paperno89a,
 Paperno89b, Paperno89c, Paperno89d, Paperno89e, Podolskaya60, Rabinovich57,
 Rabinovich58, Rabinovich60, Rabinovich61, Rabinovich61b, Savshinsky61,
 Sofronitsky46, Sofronitsky61}~;
 \item\emph{autres revues en russe}.
 \citet{Badeyan03, Barenboim45, Genn08, Piskounova}~;
 \item\emph{revues en français}.
 \citet{Cochard08a, Engerer, Lompech85a, Lompech85c, Lompech86, Macassar,
 Thomas84}~;
 \item\emph{revues en anglais}.
 \citet{Bowers59, Brown14, Juban, Leikin96, Lobanov91, Paperno84, Siepmann,
 Tassie, Young91, Zaltsberg}~;
 \item\emph{revues dans d'autres langues}.
 \citet{He12, Rattalino95, Schiavon, Sofronitskaya01, Voskobojnikov90,
 Voskobojnikov09a, Voskobojnikov09b}.
\end{itemize}

\citeauthor{Leikin96} étudie la relation entre la partition imprimée et la
musique de \Scriabine{} grâce aux enregistrements que le compositeur a
laissés de quelques-unes de ses propres œuvres.
\citeauthor{Bowers59} évoque les conditions de vie et de travail au
conservatoire de Moskva à la fin des années~1950.
\citeauthor{Engerer} abordent le lien entre les Russes et le piano.

La principale revue russe, \foreignlanguage{russian}{Советская Музыка} puis
\foreignlanguage{russian}{Музыкальная академия}, est désormais numérisée et
disponible en ligne, avec~OCR, \emph{via} le lien
\href{https://mus.academy/archive}{https://mus.academy/archive}, dès son
premier numéro publié en février~1933.

\subsubsection{Émissions radiophoniques de \citeauthor{Voskobojnikov08}}

Radio Vatican a consacré une série de neuf émissions à propos du pianiste,
conçue et présentée par \VVoskobojnikov{} \citep[voir][]{Voskobojnikov08}.

\subsubsection{Publications en ligne}

Un site Internet est consacré au pianiste \citep[voir][]{Sofronitsky09}.

Quelques publications sont disponibles en particulier sur Internet~:
\citet{Avdeeva, Bykova20, Fedorovich, HLG, Lazarev20, Letourneur, Oron,
Sofronitsky10, Summers, Vitsinsky}.

Certaines contributions en russe, déjà publiées auparavant mais souvent
difficiles à trouver, ont été regroupées sur le site Internet développé par
\citet{Badeyan10a}, à savoir~:
\begin{itemize}
 \item\citet{Badeyan10b}.
 À propos de la restauration des enregistrements du pianiste au musée
 \Scriabine{}.
 Voir \citet{Badeyan02, Badeyan03, Badeyan03b, Badeyan08}~;
 \item\citet{Lebedeva}.
 Souvenirs d'une étudiante du pianiste.
 Voir \citet{Lebedeva03}~;
 \item\citet{Nikonovich10}.
 À propos de l'activité du pianiste au musée \Scriabine{}.
 Voir \citet{Nikonovich03}~;
 \item\citet{Petropavlov}.
 Voir \citet{Petropavlov03}~;
 \item\citet{Sofronitskaya10}.
 Souvenirs de la seconde épouse du pianiste.
 Voir \citet{Sofronitskaya03}~;
 \item\citet{Zolotov10}.
 Texte d'une allocution donnée au musée \Scriabine{} le~27 octobre~2001.
 Voir \citet{Zolotov03a}.
\end{itemize}

\subsubsection{Livrets d'accompagnement de disques compacts}

\paragraph{Livrets originaux de disques compacts}

Certaines publications sur disque compact sont accompagnées d'articles
substantiels dans leurs livrets~:
\begin{itemize}
 \item
 \citet{Lischke85} pour le disque Le Chant du monde LDC~278764~;
 \item
 \citet{Ledin93} pour le disque Russian Disc RD CD~15001~;
 \item
 \citet{Malik99} pour le boîtier Philips 456~\hbox{970-2}~;
 \item
 \citet{Lobanov02} pour le disque Prometheus Editions EDITION003~;
 \item
 \citet{Rueger05} pour le boîtier Melodija MEL CD~10~00747~;
 \item
 \citet{Distler05} pour le boîtier Andante AN~1190~;
 \item
 \citet{Evans08} pour le disque Arbiter ARB~157~;
 \item
 \citet{Sofronitsky08} pour le boîtier Brilliant Classics BRIL~8975~;
 \item
 \citet{Stove15} pour le boîtier Profil Hänssler DCD PH15007~;
 \item
 \citet{Mukosey15} pour le boîtier Melodija MEL CD~10~02312.
\end{itemize}

\paragraph{Livrets de disques compacts reprenant les notes de
\citeauthor{Nikonovich79}}

Rappelons que plusieurs livrets d'accompagnement de disques compacts
reprennent en tout ou en partie les notes rédigées par \citet{Nikonovich79}
pour la série des enregistrements complets du pianiste, publiée par
l'éditeur Melodija.
Il s'agit des disques compacts des éditeurs Classound, Melodija et Arbiter~;
voir page~\pageref{sssec:NotesNikonovich}.

\subsection{Critiques de disques}

Certains disques vinyles et compacts ont été critiqués dans la presse
spécialisée~; il est difficile d'établir la liste exhaustive des articles en
question, mais on peut citer par exemple~:
\begin{itemize}
 \item
 Artia (Melodija) MK-1562 \citep{Frankenstein}~;
 \item
 Le Chant du monde LDX~78764/5 \citep{Gaillard, Harrison, LaGrange85,
 Lompech85b, Ollivier, Simmons87, Vidal}~;
 \item
 Le Chant du monde LDX~78793/4 \citep{Fousnaquer86, Warrack}~;
 \item
 Melodija D00990/1 \citep{Holcman59a, Holcman59b}~;
 \item
 Melodija/Eurodisc 85993~XFK \citep{Alluard79, Thomas84}~;
 \item
 Westminster Gold Melodija (ABC~Records) WG-8358 \citep{Hall, Horowitz79,
 Kammerer, Rabinowitz79}~;
 \item
 autres disques vinyles \citep{Clark, Fleuret, Hoffele86, LaGrange72,
 Larson, Methuen, Myers80, Myers85, Myers89}~;
 \item
 Andante AN~1190 \citep{Cairns05, Morrison05, Mulbury05, Szersnovicz}~;
 \item
 Arbiter ARB~157 \citep{Cochard09, Cowan09, Harrington09a, Sanderson,
 Woolf09b}~;
 \item
 Arlecchino ARL~1 (vol.~I) \citep{Berigan, Cochard94a, Luguenot95a}~;
 \item
 Arlecchino ARL~2 (vol.~II) \citep{Berigan, Cochard94a, Luguenot95a}~;
 \item
 Arlecchino ARL~12 (vol.~III) \citep{Bosch, Cochard94b, Luguenot95a}~;
 \item
 Arlecchino ARL~28 (vol.~IV) \citep{Cochard95a, Luguenot95a}~;
 \item
 Arlecchino ARL~31 (vol.~V) \citep{Cochard95b, Luguenot95a}~;
 \item
 Arlecchino ARL~39 (vol.~VI) \citep{Cochard95a, Luguenot95a}~;
 \item
 Arlecchino ARL~40 (vol.~VII) \citep{Cochard95a, Luguenot95a}~;
 \item
 Arlecchino ARL~41 (vol.~IX) \citep{Luguenot95a}~;
 \item
 Arlecchino ARL~42 (vol.~VIII) \citep{Luguenot95a}~;
 \item
 Arlecchino ARL~61 (vol.~X) \citep{Luguenot95a}~;
 \item
 Arlecchino ARL~62 (vol.~XI) \citep{Luguenot95a}~;
 \item
 Arlecchino ARL~67 (vol.~XII) \citep{Luguenot95a}~;
 \item
 Arlecchino ARL~95 (vol.~XIII) \citep{Luguenot95b, Schonberg95a}~;
 \item
 Arlecchino ARL~119 (vol.~XIV) \citep{Schonberg95a}~;
 \item
 Arlecchino ARL~155 (vol.~XVI) \citep{Bosch, Schonberg96}~;
 \item
 Arlecchino ARL~174 (vol.~XVII) \citep{Hawkins96}~;
 \item
 Arlecchino ARL~183 (vol.~XVIII) \citep{Hawkins96}~;
 \item
 Brilliant Classics BRIL~8975 \citep{Clarke, Cochard08b, Cowan08,
 Harrington09b, Haylock09, Naulleau11, Woolf09a}~;
 \item
 Brilliant Classics BRIL~9014 \citep{Baron09, Muhlbock}~;
 \item
 Brilliant Classics BRIL~9241 \citep{Harrington12}~;
 \item
 Le Chant du monde LDC~278764 \citep{Baron13, Boissard13, Rabinowitz89,
 Stewart}~;
 \item
 Le Chant du monde LDC~278765 \citep{Rabinowitz89, Sanders88, Simmons,
 Stewart, Sykora88}~;
 \item
 Le Chant du monde LDC~288032 \citep{Freslon92, Hawkins92, McLachlan92,
 Rabinowitz92, Rabinowitz98, Stewart}~;
 \item
 Classical Records CR-004 \citep{Woolf06}~;
 \item
 Classical Records CR-014 \citep{Leonard}~;
 \item
 Denon COCQ-84242 \citep{Marcinik}~;
 \item
 Diapason (Les Indispensables \Number{58}) DIAP058 \citep{Boissard14a}~;
 \item
 Harmonia Mundi HMC~1905169 \citep{Drillon, Fanning87, Glass, LaGrange86,
 Moor87, Roux86, Wiser}~;
 \item
 Melodija MEL CD~74321~25177-2 \citep{Bernager, Fanning95, Layton,
 Pappastavrou96, Schonberg95b}~;
 \item
 Melodija MEL CD~10~02237 \citep{Arloff, Boissard14b, Cowan15,
 Harrington15a}~;
 \item
 Melodija MEL CD~10~02312 \citep{Baronian, Becker16, Boissard16, Cowan16,
 Duerer, Hoffele16, Theurich, Westbrook}~;
 \item
 Multisonic Russian Treasure MT~310181-2 \citep{Fanning94, Hoffele,
 Schonberg01}~;
 \item
 Palladio PD~4131 \citep{Evans94}~;
 \item
 Philips 456~\hbox{970-2} \citep{Johnson, Moreau, Szersnovicz99,
 VincentLancrin, Young99}~;
 \item
 Profil Hänssler DCD PH15007 \citep{Beyne, Bigorie, Boissard15, Church,
 Greenbank, Harrington15b, Hulot}~;
 \item
 Prometheus Editions EDITION003 \citep{Fanning03, Morrison02, Potter,
 Sturrock}~;
 \item
 Russian Compact Disc RCD~16289 \citep{Masters19}~;
 \item
 Russian Disc RD CD~15001 \citep{Evans93, Hoffele}~;
 \item
 Scribendum SC817 \citep{Chizynski, Distler20, Hoffele20}~;
 \item
 Selene \hbox{CD-s}~9809.45 \citep{Rabinowitz99}~;
 \item
 Urania SP~4205 \citep{Danzin, Lompech03}~;
 \item
 Urania SP~4215 \citep{Bolen}~;
 \item
 Urania URN~22.299 \citep{DiazGomez}~;
 \item
 Vista Vera VVCD-00024 \citep{Marcinik, Woolf04a}~;
 \item
 Vista Vera VVCD-00031 \citep{Woolf04b}~;
 \item
 Vista Vera VVCD-00093 \citep{Harrington07, Lewis}~;
 \item
 Vista Vera VVCD-00224 \citep{Woolf11a}~;
 \item
 Vista Vera VVCD-00225 \citep{Woolf11b}.
\end{itemize}

\section[%
Résumé du répertoire enregistré de Vladimir Sofronickij]{%
Résumé du répertoire enregistré de \VSofronitsky{}}

\InputIfFileExists{dvs.res}{}{}

\section{Quelques informations à propos de l'éditeur Melodija}

Les informations de cette section proviennent du document rédigé par
\citet{Crocker} et de la page \citet{WikipediaMelodiya} relative à Melodija.
Le site \citet{Rockdisco} fournit les séquences de numéros de catalogue pour
chaque année de production.
\citet{Bennett} est une référence pour les publications soviétiques sur
disque vinyle de~1951 aux années~1970.

Toute l'industrie du disque en~URSS est restée un monopole d'état de~1919
jusqu'à la fin des années~1980~; l'éditeur Melodija, quant à lui, a été
introduit en~1964.
Par souci de simplification, les parutions en~URSS de disques 78~tours et de
disques vinyles de longue durée antérieures à l'introduction de Melodija
sont ici attribuées de même à l'éditeur Melodija.
En raison de son appartenance à l'état, l'industrie du disque en~URSS a pu
appliquer un système unifié de numérotation de ses parutions, quelle que
soit l'origine de l'enregistrement et quel que soit le lieu de production,
et ce dès~1933.
Pour les disques 78~tours et pour les disques vinyles de longue durée, les
séquences de numéros de catalogue sont chronologiques, ce qui permet ainsi
de dater la plupart des parutions grâce à leurs numéros de catalogue.

À partir de~1933 et à l'époque des disques 78~tours, un système unifié de
numérotation a donc été adopté~; chaque face de disque porte un numéro
unique, précédé du chiffre~0 dans le cas des disques de~30\,cm, et les
nombres suivent une séquence chronologique (en faisant abstraction du zéro
en première position s'il y en a un).
Les deux faces d'un disque portent des numéros consécutifs, sauf dans le cas
de nouveaux couplages (chaque face conserve alors son numéro original).
On connaît les séquences de numéros pour chaque année de production, mais il
ne s'agit pas d'une manière nécessaire de l'année d'enregistrement.
De nouveaux disques 78~tours ont encore été produits bien après l'arrivée
des premiers disques vinyles de longue durée.

Les disques vinyles de longue durée ont été introduits en~1951.
Leur système de numérotation est semblable à celui des disques 78~tours, en
commençant par le chiffre~1~; les numéros sont précédés de la lettre
cyrillique~\foreignlanguage{russian}{Д}, \cad \foreignlanguage{russian}
{долго-играющи} (en français, \emph{longue durée}) -- à nouveau, chaque face
de disque porte un numéro unique.
Un numéro sans préfixe désigne un disque de~25\,cm, un zéro unique désigne
un disque de~30\,cm, deux zéros désignent un disque de~20\,cm et trois zéros
désignent un disque de~17,5\,cm.
Une nouvelle séquence chronologique de numéros a été initiée en~1961 pour
les disques enregistrés en stéréophonie~; ils sont précédés de la lettre
cyrillique~\foreignlanguage{russian}{С} -- système stéréophonique -- ou des
deux lettres cyrilliques~\foreignlanguage{russian}{СМ} -- système compatible
stéréophonique et monophonique.
De nouveaux disques monophoniques ont encore été produits bien après
l'introduction de la stéréophonie~; étant donné que les deux séquences sont
chronologiques, les numéros de catalogue dans les
séries~\foreignlanguage{russian}{Д} et~\foreignlanguage{russian}{С}
diffèrent.
Le nombre~33, qui peut apparaître en préfixe de la lettre
cyrillique~\foreignlanguage{russian}{Д} ou~\foreignlanguage{russian}{С}, est
utilisé pour différencier le disque en question des disques microsillons
45~tours ou 78~tours.
La lettre cyrillique~\foreignlanguage{russian}{Н} juste avant la lettre
cyrillique~\foreignlanguage{russian}{Д} signale un disque reproduit -- le
numéro original est alors conservé.

En juillet~1975, l'utilisation des zéros a été abandonnée au profit du code
\foreignlanguage{russian}{С}XX{-}YYYYY -- cas stéréophonique -- ou
\foreignlanguage{russian}{М}XX{-}YYYYY -- cas monophonique~; le nombre~XX
représente le type de musique enregistrée (premier chiffre~X) et la
dimension du disque (second chiffre~X), tandis que le nombre~YYYYY est le
numéro de face.
Les séquences numériques recommencent là où l'ancien système s'était arrêté
et le nombre~XX peut donc être négligé dans l'optique d'une datation.
Lors des productions ultérieures, le numéro de catalogue est suivi d'un
nombre à trois chiffres compris entre~001 et~009~; la signification de ce
code supplémentaire est mystérieuse.
Après~1975, les rééditions d'enregistrements anciens, souvent monophoniques,
ont été numérotées à nouveau, mais cette fois selon le nouveau système, ce
qui induit un abandon du système de numérotation chronologique.

Les enregistrements effectués par Melodija en~URSS et importés dans d'autres
pays y sont souvent vendus sous le label~MK, \cad \foreignlanguage{russian}
{Международная Книга} (en français, \emph{livre international})~; ces
enregistrements ont été diffusés pour le marché international.
Le label Russian Disc est un de ceux sous lesquels certaines usines ont
continué à produire des disques vinyles de longue durée après~1991.
Le label Akkord date du début des années~1960~; on pense qu'il a été utilisé
par la seule usine de Leningrad.
Il a disparu de manière progressive après l'introduction de Melodija.

\section{Série des \Quote{Enregistrements complets} éditée par Melodija}

La série de sept volumes publiés des \Quote{Enregistrements complets}, sur
un total de douze volumes prévus, réalisée par la firme Melodija, constitue
l'édition fondamentale.
Rappelons que les volumes~1 à~4 et le volume~11 de cette série inachevée
n'ont jamais été publiés à l'époque du disque vinyle et qu'ils contiennent
de nombreux enregistrements qui n'ont pas été publiés avant longtemps
\citep[voir][p.~64]{Malik}.
Le volume~1 constitue cependant un cas particulier, puisqu'il dispose d'un
numéro de catalogue (M10 49083/92) et que son contenu est référencé.
Selon \INikonovich{} \citep[voir][p.~11]{Nikonovich11}, traduit en italien
par \ARossi{} \citep[voir][]{Rossi}, la publication a cessé, après le
volume~5 ou~10, publié en dernier lieu, en raison de malentendus entre les
personnes qui prenaient part au projet de publication des enregistrements
complets de \VSofronitsky{}, et ce bien que le volume suivant, sans doute le
volume~1, ait été préparé.
C'était en outre peu de temps avant l'émergence de la technologie du disque
compact.
On trouve en effet, de manière très sporadique, des mentions du volume~1 de
la série~: voir \citet{Recordssu} et \citet{Rockdisco}~; il s'agit d'un
boîtier de cinq disques vinyles dont les exemplaires n'apparaissent pourtant
jamais sur les sites de vente de seconde main.
Il semble que la série de douze volumes ait été prévue, à l'origine, selon
la disposition suivante -- voir \citet[message de densen2002 daté 2006-09-26
à~21\up{h}~29\up{m}]{Forumklassika.ru} et aussi \citet[p.~94-107]{White}~:
\begin{itemize}
 \item \Volume{1} et \Volume{2} (jamais publiés)~:
 enregistrements d'archives principaux réalisés à partir des retransmissions
 sonores effectuées au conservatoire d'\hbox{État} de Moskva~;
 \item \Volume{3} et \Volume{4} (jamais publiés)~:
 enregistrements effectués en récital à \MSHM~;
 \item \Volume{5}~:
 enregistrements effectués en récital à \LPGH~;
 \item \Volume{6}~:
 enregistrements effectués en récital à \Moscow entre~1949 et~1955~;
 \item \Volume{7} et \Volume{8}~:
 enregistrements effectués lors de six récitals de \VSofronitsky{} à \MCSH
 entre~1958 et~1960~;
 \item \Volume{9} à \Volume{12} (\Volume{11} jamais publié)~:
 enregistrements principaux du fonds de diffusions radiophoniques d'URSS et
 disques d'enregistrements en studio -- recensement des années~1930 aux
 années~1950.
\end{itemize}

Le \Volume{5} (M10 45223/32 005) a été publié en~1983, le \Volume{6} (M10
42469/78) en~1980, le \Volume{7} (M10 42795/804) en~1980, le \Volume{8} (M10
42253/64) en~1979, le \Volume{9} (M10 44603/14) en~1982, le \Volume{10} (M10
45937/48 009) en~1984 et le \Volume{12} (M10 43119/30) en~1981
\citep[voir][]{Rockdisco}.

Le \Volume{1} (M10 49083/92) est un cas particulier, comme indiqué plus
haut.
Selon les bases de données \citet{Recordssu} et \citet{Rockdisco} -- seules
références qui évoquent ce \Quote{premier} volume de la série --, il s'agit
de trois récitals donnés à la grande salle du conservatoire de Moskva
en~1948, 1951 et~1952, et ce volume aurait été publié beaucoup plus tard que
les sept précédents, en~1990.
Il serait composé des récitals ou fragments de récitals suivants~:
\begin{itemize}
 \item
 récital daté~1948-01-05~: \Schubert{} (Impromptu D~899 \Number{3} et
 Impromptu D~899 \Number{4}).
 \Schumann{} (Sonate \Opus{11}).
 \Chopin{} (Polonaise \Opus{44}).
 \Liszt{} (Sonate S~178).
 \Schubert{}/\Liszt{} (\emph{Der Müller und der Bach}, \emph{Der
 Doppelgänger}, \emph{Auf dem Wasser zu singen} et \emph{Erlkönig}).
 Voir Vista Vera VVCD-00203 (2009) et Vista Vera VVCD-00204 (2009) pour
 l'édition sur disque compact~;
 \item
 fragment du récital daté~1952-10-10~: \Beethoven{} (Sonate \Opus{57}).
 \Schumann{} (Carnaval \Opus{9}).
 \Rachmaninov{} (Étude-tableau \Opus{39} \Number{4} et Étude-tableau
 \Opus{39} \Number{6}).
 \Prokofiev{} (Pièce \Opus{12} \Number{3} et Pièce \Opus{12} \Number{7}).
 \Rachmaninov{} (\emph{Oriental Sketch} et Étude-tableau \Opus{33}
 \Number{6}).
 Voir Moscow State Conservatoire SMC CD~0020 (1998) pour l'édition sur
 disque compact~;
 \item
 fragment du récital daté~1951-11-26~: \Mozart{} (Fantaisie K~396).
 \Schumann{} (Fantaisie \Opus{17}).
 \Liszt{} (Après une lecture de Dante~: \emph{Fantasia quasi Sonata} S~161
 \Number{7}).
 \Rachmaninov{} (Moment musical \Opus{16} \Number{5} et Moment musical
 \Opus{16} \Number{2}).
 \Scriabine{} (Sonate \Opus{30}).
 \Debussy{} (\emph{Serenade of the Doll} L~113 \Number{III}).
 \Liadov{} (\emph{A Musical Snuffbox} \Opus{32}).
 \Prokofiev{} (Sarcasme \Opus{17} \Number{5}).
 Voir Moscow State Conservatoire SMC CD~0019 (1998) pour l'édition sur
 disque compact.
\end{itemize}

Pour les enregistrements non compris dans les sept volumes publiés de cette
série -- \Volume{5}~: M10 45223/32 005~; \Volume{6}~: M10 42469/78~;
\Volume{7}~: M10 42795/804~; \Volume{8}~: M10 42253/64~; \Volume{9}~: M10
44603/14~; \Volume{10}~: M10 45937/48 009~; \Volume{12}~: M10 43119/30 --,
la présente liste indique le numéro du disque vinyle d'origine, Melodija ou
autre, si l'enregistrement en question a en effet fait l'objet d'une édition
sur ce support \citep[voir][]{Malik, Masuda, Nikonovich11, Rossi, White}.
C'est la raison pour laquelle la discographie ne contient jamais plus d'une
référence sur disque vinyle pour chaque interprétation~; la discographie ne
prétend donc pas à l'exhaustivité dans le cas des parutions sur disque
vinyle.
Néanmoins, les informations très détaillées relatives aux disques vinyles
\Quote{isolés}, qui figurent dans la thèse de~DMA d'\EWhite{}
\citep[voir][p.~83-94]{White} et dans le document d'\INikonovich{}
\citep[voir][p.~\hbox{1-5}]{Nikonovich11}, ont été reprises aux sections
suivantes.

La maison d'édition Melodija a en outre publié de nombreux disques vinyles
\Quote{isolés}, consacrés à \VSofronitsky{} \citep[voir][]{Malik, Masuda,
Nikonovich11, Recordssu, Rockdisco, Rossi, White}.
Quelques autres maisons d'édition ont aussi publié un nombre restreint de
disques vinyles supplémentaires.

Tous les originaux des enregistrements du pianiste sont conservés dans les
archives de la radio-télévision Gosteleradio de Moskva et de
Sankt-Peterburg, dans celles de la firme Melodija, dans celles du
conservatoire d'\hbox{État} de Moskva, et dans celles du musée \Scriabine{}
situé dans le quartier d'\hbox{Arbat} à Moskva
\citep[voir][p.~11]{Nikonovich11}.

\section{Disques 78~tours édités par Melodija}

On peut aussi consulter l'encyclopédie de musique enregistrée établie par
\citet{CloughCuming}, seule source qui mentionne quelques-uns de ces disques
78~tours en Occident.

\subsection{5655/6 (78~tours)}

\PublishDate{1937}
\Chopin{} (Mazurka \Opus{41} \Number{2} et Étude \Opus{10} \Number{4}
[1937-06-16]).
Ce disque~78~tours n'est mentionné que dans la discographie établie par
\citet[p.~1]{Nikonovich11}.
Il est repris dans celle due à \citet[p.~65]{Malik}.
Voir aussi \citet[p.~375]{Scriabine}.

\subsection{5690/5695 (78~tours)}

\PublishDate{1937}
\Scriabine{} (Étude \Opus{8} \Number{10} [1937-06-25]).
\Liszt{} (Gnomenreigen S~145 \Number{2} [1937-06-25]).

\subsection{6837/8 (78~tours)}

\PublishDate{1938}
\Scriabine{} (Poème \Opus{32} \Number{1} et Préludes \Opus{27} \Number{1} et
\Opus{27} \Number{2} [1938-03-28]).

\subsection{7659/60 (78~tours)}

\PublishDate{1938}
\Chopin{} (Mazurkas \Opus{41} \Number{2} et \Opus{63} \Number{2}
[1938-11-17]).

\subsection{07661/2 (78~tours)}

\PublishDate{1938}
\Goltz{} (Scherzo en \kE mineur et Prélude en \kE mineur [1938-11-17]).

\subsection{10857/8 (78~tours)}

\PublishDate{1941}
\Chopin{} (Valses \Opus{64} \Number{1} et \Opus{70} \Number{1} [1941]).

\subsection{10958/9 (78~tours)}

\PublishDate{1941}
\Schumann{} (Bunte Blätter \Opus{99} \Number{1}, \Opus{99} \Number{6} et
\Opus{99} \Number{7} [1941]).

\subsection{12438/9 (78~tours)}

\PublishDate{1945}
\Scriabine{} (Étude \Opus{8} \Number{11} [1945-01-16]).

\subsection{12440/12449 (78~tours)}

\PublishDate{1945}
\Rachmaninov{} (Préludes \Opus{32} \Number{5} [1945-01-16] et \Opus{32}
\Number{12} [1945-01-18]).

\subsection{12450/1 (78~tours)}

\PublishDate{1945}
\Scriabine{} (Étude \Opus{8} \Number{9} [1945-01-18]).
Selon \FMalik{} \citep[voir][p.~72]{Malik}, la date de cet enregistrement
est 1945-01-30.

\subsection{13076/7 (78~tours)}

\PublishDate{1945}
\Chopin{} (Valses \Opus{64} \Number{3} et \Opus{69} \Number{1}
[1945-05-22]).

\subsection{13081/2 (78~tours)}

\PublishDate{1945}
\Scriabine{} (Étude \Opus{8} \Number{12} [1945-08-28] et Préludes \Opus{13}
\Number{6} et \Opus{31} \Number{3} [1945-08-27]).
Selon \FMalik{} \citep[voir][p.~74]{Malik}, la date d'enregistrement des
deux Préludes est 1945-08-28 comme pour l'\hbox{Étude}.

\subsection{13093/4 (78~tours)}

\PublishDate{1945}
\Scriabine{} (Étude \Opus{8} \Number{10} et Préludes \Opus{11} \Number{14}
et \Opus{17} \Number{4} [1945-08-28]).

\subsection{14917/8 (78~tours)}

\PublishDate{1947}
\Prokofiev{} (Conte de la vieille grand-mère \Opus{31} \Number{3}
[1947-07-02]).
\Medtner{} (Conte de fées (Skazka) \Opus{20} \Number{2} [1947-07-02]).

\subsection{015000/1 (78~tours)}

\PublishDate{1947}
\Rachmaninov{} (Moment musical \Opus{16} \Number{3} et Prélude \Opus{23}
\Number{4} [1947-07-09]).

\subsection{15145/6 (78~tours)}

\PublishDate{1947}
\Chopin{} (Mazurka \Opus{50} \Number{3} [1947-09-13]).

\subsection{15533/4 (78~tours)}

\PublishDate{1948}
\Chopin{} (Polonaise \Opus{26} \Number{1} [1948-01-13]).

\subsection{15861/2 (78~tours)}

\PublishDate{1948}
\Scriabine{} (Valse \Opus{38} [1948-04-14]).

\subsection{17061/2 (78~tours)}

\PublishDate{1949}
\Chopin{} (Prélude \Opus{28} \Number{15} [1949-06-14]).

\subsection{17063/4 (78~tours)}

\PublishDate{1949}
\Liadov{} (Prélude \Opus{11} \Number{1} et Une Tabatière à musique \Opus{32}
[1949-06-14]).

\subsection{17186/7 (78~tours)}

\PublishDate{1949}
\Liadov{} (Prélude \Opus{57} \Number{1} et Valse \Opus{57} \Number{2}
[1949-07-15]).

\subsection{17188/9 (78~tours)}

\PublishDate{1949}
\Liadov{} (Novelette \Opus{20}, Prélude \Opus{31} \Number{2} et Mazurka
\Opus{57} \Number{3} [1949-07-15]).

\subsection{18367/8 (78~tours)}

\PublishDate{1950}
\Chopin{} (Mazurka \Opus{41} \Number{1} et Valse \Opus{70} \Number{2}
[1950-07-21]).

\subsection{18627/8 (78~tours)}

\PublishDate{1950}
\Liszt{} (\emph{Canzonetta del Salvator Rosa} S~161 \Number{3}
[1949-03-09]).
Ce disque n'est pas mentionné par \INikonovich{}
\citep[voir][]{Nikonovich11} et seule sa face~B (18628) est consacrée à
\VSofronitsky{}.
Voir \citet{Recordssu}.

\subsection{18954/5 (78~tours)}

\PublishDate{1950}
\Scriabine{} (Polonaise \Opus{21} [1950-11-03]).
Disque mentionné dans l'encyclopédie établie par \citet[supplément~II,
p.~203]{CloughCuming}.

\subsection{19583/4 (78~tours)}

\PublishDate{1951}
\Scriabine{} (Préludes \Opus{11} \Number{2} et \Opus{11} \Number{4} [1951]).
Disque mentionné dans l'encyclopédie établie par \citet[supplément~II,
p.~203]{CloughCuming}.

\subsection{19585/6 (78~tours)}

\PublishDate{1951}
\Scriabine{} (Préludes \Opus{11} \Number{5}, \Opus{11} \Number{17} et
\Opus{11} \Number{20} [1951]).
Disque mentionné dans l'encyclopédie établie par \citet[supplément~II,
p.~203]{CloughCuming}.

\subsection{19975/6 (78~tours)}

\PublishDate{1951}
\Scriabine{} (Préludes \Opus{11} \Number{7}, \Opus{11} \Number{8}, \Opus{11}
\Number{19} et \Opus{11} \Number{22} [1951]).
Ce disque n'est pas mentionné par \INikonovich{}
\citep[voir][]{Nikonovich11}, mais il figure dans l'encyclopédie de
\citet[supplément~II, p.~203]{CloughCuming}.

\subsection{20974/5 (78~tours)}

\PublishDate{1951}
\Scriabine{} (Préludes \Opus{39} \Number{2}, \Opus{48} \Number{2} et
\Opus{31} \Number{4}).
\EWhite{} \citep[voir][p.~83]{White} ne donne pas la date de ces
enregistrements.
Ce disque n'est pas mentionné par \INikonovich{}
\citep[voir][]{Nikonovich11}, mais il figure dans l'encyclopédie de
\citet[supplément~II, p.~203]{CloughCuming}.

\subsection{21854/5 (78~tours)}

\PublishDate{1952}
\Glazounov{} (Prélude \Opus{49} \Number{1} [1952-07-01]).
\Scriabine{} (Prélude \Opus{35} \Number{2}).
Disque mentionné dans l'encyclopédie établie par \citet[supplément~III,
p.~177]{CloughCuming}.

\subsection{D~867/8 (78~tours)}

\PublishDate{1952}
\Scriabine{} (Préludes \Opus{11} \Number{1} à \Number{12}).
Ce disque n'est pas mentionné par \INikonovich{}
\citep[voir][]{Nikonovich11}.
Voir \citet{Recordssu}.

\subsection{D~953/4 (78~tours)}

\PublishDate{1952}
\Chopin{} (Préludes \Opus{28} \Number{1} à \Number{12}).
Ce disque n'est pas mentionné par \INikonovich{}
\citep[voir][]{Nikonovich11}.
Voir \citet{Recordssu}.

\subsection{00995/6 (78~tours)}

\PublishDate{1952}
\Chopin{} (Préludes \Opus{28} \Number{13}, \Opus{28} \Number{14} et
\Opus{28} \Number{15}).
Ce disque n'est pas mentionné par \INikonovich{}
\citep[voir][]{Nikonovich11}.

\subsection{001041/2 (78~tours)}

\PublishDate{1952}
\Chopin{} (Préludes \Opus{28} \Number{17}, \Opus{28} \Number{18}, \Opus{28}
\Number{19}, \Opus{28} \Number{20}, \Opus{28} \Number{21}, \Opus{28}
\Number{22} et \Opus{28} \Number{23}).
Ce disque n'est pas mentionné par \INikonovich{}
\citep[voir][]{Nikonovich11}.

\subsection{1225/6 (78~tours)}

\PublishDate{1953}
\Liszt{} (Sonetto~123 del Petrarca S~161 \Number{6} [1952-07-23]).
\Schumann{} (Arabesque \Opus{18} [1952-07-23]).
Ce disque n'est pas mentionné par \INikonovich{}
\citep[voir][]{Nikonovich11}.

\subsection{1255/6 (78~tours)}

\PublishDate{1953}
\Scriabine{} (Polonaise \Opus{21} [1950-11-03], Impromptu \Opus{14}
\Number{2} [1948] et Mazurka \Opus{25} \Number{3} [1948-07-06
ou~1952-01-30]).
Ce disque n'est pas mentionné par \INikonovich{}
\citep[voir][]{Nikonovich11}.
Voir \citet{Recordssu}.

\subsection{23052/3 (78~tours)}

\PublishDate{1953}
\Scriabine{} (Préludes \Opus{16} \Number{3} [1952] et \Opus{35} \Number{2}
[1948 ou~1948 à~1953]).
Ce disque n'est pas mentionné par \INikonovich{}
\citep[voir][]{Nikonovich11}.
Voir \citet{Recordssu}.

\subsection{32150/1 (78~tours)}

\PublishDate{1959}
\Scriabine{} (Poèmes \Opus{32} \Number{1} et \Opus{32} \Number{2}).
Ce disque n'est pas mentionné par \INikonovich{}
\citep[voir][]{Nikonovich11} et les dates d'enregistrement sont inconnues.
Voir \citet{Recordssu}.

\section{Disques vinyles de longue durée édités par Melodija}

\subsection{Melodija D00990/1}

\PublishDate{1952}
\Chopin{} (Préludes \Opus{28}).
\EWhite{} \citep[voir][p.~83]{White} spécifie la date d'enregistrement
\Quote{environ~1958}, mais cela semble exclu pour les Préludes \Opus{28}
selon \FMalik{} \citep[voir][p.~66]{Malik}.

\subsection{Melodija D04888/9}

\PublishDate{1959}
\Scriabine{} (Sonate \Opus{23}, Sonate \Opus{68}, Deux Danses \Opus{73},
Poème \Opus{72}).
Enregistrements en~1958 et en~1959.
Ce disque a été publié du vivant de \VSofronitsky{}.

\subsection{Melodija D06937/8}

\PublishDate{1960}
\Scriabine{} (Seize Préludes \Opus{13} \Number{1}, \Opus{11} \Number{2},
\Opus{13} \Number{3}, \Opus{11} \Number{5}, \Opus{15} \Number{1}, \Opus{11}
\Number{9}, \Opus{22} \Number{2}, \Opus{22} \Number{3}, \Opus{16}
\Number{2}, \Opus{16} \Number{3}, \Opus{16} \Number{4}, \Opus{11}
\Number{15}, \Opus{11} \Number{16}, \Opus{11} \Number{21}, \Opus{11}
\Number{22} et \Opus{11} \Number{24}, Cinq Poèmes \Opus{52} \Number{1},
\Opus{59} \Number{1}, \Opus{51} \Number{3}, \Opus{52} \Number{3} et
\Opus{36}, Sonate \Opus{68}, Morceau \Opus{51} \Number{1}, Morceau \Opus{45}
\Number{1}, Mazurka \Opus{25} \Number{3}, Étude \Opus{42} \Number{5}).
Enregistrements 1960-02-02.
Ce disque a été publié du vivant de \VSofronitsky{}.
Ce disque a été distribué pour l'exportation, en~1978, sous le label
Westminster Gold Melodija (ABC~Records) WG-8358 \citep[voir][p.~83]{White}.

\subsection{Melodija D08723/4}

\PublishDate{1961}
\Chopin{} (Deux Nocturnes \Opus{27} \Number{1} et \Opus{27} \Number{2}~;
Deux Valses \Opus{70} \Number{2} et \Opus{70} \Number{3}~; Impromptu
\Opus{51}~; Cinq Mazurkas \Opus{30} \Number{2}, \Opus{33} \Number{3},
\Opus{33} \Number{4}, \Opus{41} \Number{2} et \Opus{68} \Number{4}~;
Barcarolle \Opus{60}).
Enregistrements en~1960.

\subsection{Melodija D08725/6}

\PublishDate{1961}
\Schubert{} (Cinq Moments musicaux D~780 \Number{1-4} et \Number{6} [1959],
Trois Impromptus D~899 \Number{1} [1953], D~899 \Number{3} [1960], D~935
\Number{2} [1960]).
\INikonovich{} \citep[voir][p.~2]{Nikonovich11} donne une date
d'enregistrement en~1959 pour les deux derniers Impromptus, mais cela est
contredit par \FMalik{} \citep[voir][p.~70]{Malik}.

\subsection{Melodija D08763/4}

\PublishDate{1961}
\Schumann{} (Papillons \Opus{2} [1952], Bunte Blätter \Opus{99}
\Number{1-8} [1953], Arabesque \Opus{18} [1952], Deux Novelettes \Opus{21}
\Number{1} [1953] et \Opus{21} \Number{8} [1953]).

\subsection{Melodija D08779/80}

\PublishDate{1961}
\Scriabine{} (Cinq Études \Opus{8} \Number{2} [1959], \Opus{8} \Number{4}
[1959], \Opus{8} \Number{5} [1959], \Opus{8} \Number{9} [1959] et \Opus{8}
\Number{11} [1959], Fantaisie \Opus{28} [1959], Sonate \Opus{70}
[1960-01-08], Trois Préludes \Opus{11} \Number{16} [1958], \Opus{35}
\Number{2} [1958] et \Opus{37} \Number{1} [1958], Deux Poèmes \Opus{32}
\Number{1} [1958] et \Opus{32} \Number{2} [1958]).
\INikonovich{} \citep[voir][p.~2]{Nikonovich11} donne une date
d'enregistrement en~1959 pour les trois Préludes et pour les deux Poèmes,
mais cela est contredit par \FMalik{} \citep[voir][p.~72 et p.~74]{Malik}.

\subsection{Melodija D8885/6}

\PublishDate{1961}
\Scriabine{} (Deux Mazurkas \Opus{3} \Number{6} [1952] et \Opus{3}
\Number{9} [1952], Morceau \Opus{2} \Number{3} [1953], Deux Impromptus
\Opus{12} \Number{2} [1950] et \Opus{14} \Number{2} [1948]).

\subsection{Melodija D9047/8}

\PublishDate{1961}
\Prokofiev{} (Contes de la vieille grand-mère \Opus{31} [1946], Vision
fugitive \Opus{22} \Number{7} [1946], Sarcasme \Opus{17} \Number{3} [1946],
Six Pièces \Opus{12} \Number{2} [1953], \Opus{12} \Number{3} [1953],
\Opus{12} \Number{6} [1953], \Opus{12} \Number{7} [1953], \Opus{12}
\Number{8} [1953] et \Opus{12} \Number{9} [1953]).
Ce disque a été publié à nouveau sous le numéro de catalogue D016161/2.

\subsection{Melodija D09085/6}

\PublishDate{1961}
\Chopin{} (Trois Études \Opus{10} \Number{3} [1953], \Opus{10} \Number{6}
[1948] et Nouvelle Étude \Number{2} en \kA \Flat majeur [1948], Scherzo
\Opus{20} [1952], Six Mazurkas \Opus{41} \Number{1}, \Opus{17} \Number{3},
\Opus{24} \Number{1}, \Opus{50} \Number{3}, \Opus{68} \Number{2} et
\Opus{68} \Number{3} [1947-1953]).

\subsection{Melodija D09145/6}

\PublishDate{1961}
\Beethoven{} (Sonate \Opus{28} [1953] et Andante favori WoO~57 [1952]).
\Mendelssohn{} (Variations sérieuses \Opus{54} [1950]).

\subsection{Melodija D09155/6 ou Melodija D09156/7}

\PublishDate{1961}
Le numéro de catalogue est D09155/6 selon \citet{Nikonovich11} et
\citet{White}, et D09156/7 selon \citet{Masuda}.
\Scriabine{} (Trois Morceaux \Opus{2} \Number{1} [1949], \Opus{2} \Number{2}
[1953] et \Opus{2} \Number{3} [1953], Quatre Mazurkas \Opus{25} \Number{3}
[1948], \Opus{25} \Number{7} [1953], \Opus{25} \Number{8} [1952] et
\Opus{40} \Number{2} [1948], Polonaise \Opus{21} [1950], Prélude \Opus{9}
\Number{1} [1948], Étude \Opus{8} \Number{1} [1950], Deux Impromptus
\Opus{12} \Number{2} [1950] et \Opus{14} \Number{2} [1948], Quatre Préludes
\Opus{31} \Number{1} [1951], \Opus{31} \Number{2} [1953], \Opus{31}
\Number{3} [1946] et \Opus{31} \Number{4} [1953], Morceau \Opus{56}
\Number{2} [1946], Poème \Opus{44} \Number{2} [1953]).
\INikonovich{} \citep[voir][p.~2]{Nikonovich11} donne une date
d'enregistrement en~1953 pour le Morceau \Opus{2} \Number{1} (Étude), mais
cela est contredit par \FMalik{} \citep[voir][p.~72]{Malik}.

\subsection{Melodija D09405/6}

\PublishDate{1962}
\Rachmaninov{} (Sept Préludes \Opus{3} \Number{2} [1949], \Opus{23}
\Number{1} [1946], \Opus{23} \Number{4} [1946], \Opus{23} \Number{6} [1946],
\Opus{32} \Number{3} [1946], \Opus{32} \Number{5} [1946] et \Opus{32}
\Number{12} [1946], Deux Moments musicaux \Opus{16} \Number{5} [1946] et
\Opus{16} \Number{3} [1952], Cinq Études-tableaux \Opus{39} \Number{4}
[1953], \Opus{33} \Number{2} [1951], \Opus{33} \Number{7} [1946], \Opus{39}
\Number{5} [1951] et \Opus{39} \Number{6} [1946]).

\subsection{Melodija ND09405/6}

\PublishDate{1962}
Ce disque est en partie différent du précédent.
\Rachmaninov{} (Sept Préludes \Opus{3} \Number{2} [1949], \Opus{23}
\Number{1} [1946], \Opus{23} \Number{4} [1946], \Opus{23} \Number{6} [1946],
\Opus{32} \Number{3} [1946], \Opus{32} \Number{5} [1960-10-28] et \Opus{32}
\Number{12} [1960-10-28], Trois Moments musicaux \Opus{16} \Number{5}
[1960-05-13], \Opus{16} \Number{2} [1960-05-13] et \Opus{16} \Number{3}
[1952], Cinq Études-tableaux \Opus{39} \Number{4} [1953], \Opus{33}
\Number{7} [1946], \Opus{33} \Number{2} [1951], \Opus{39} \Number{5} [1951]
et \Opus{39} \Number{6} [1946]).

\subsection{Melodija D010315/6}

\PublishDate{1962}
\Schubert{}/\Liszt{} (Frühlingsglaube, Litanei, Der Müller und der Bach, Der
Doppelgänger, Aufenthalt).
\Liszt{} (Funérailles S~173 \Number{7}, Sonetto~104 del Petrarca S~161
\Number{5}, Valse oubliée \Number{1} S~215 \Number{1}).
Enregistrements en octobre~1960.
\INikonovich{} \citep[voir][p.~2]{Nikonovich11} donne la date
d'enregistrement 1960-10-22/28 \citep[voir aussi][p.~70]{White}.

\subsection{Melodija D010353/4}

\PublishDate{1962}
\Chopin{} (Préludes \Opus{28} \Number{1-15}, \Number{17-20}, \Number{22} et
\Number{23} [1950-1953], Prélude \Opus{45} [1950]).
\INikonovich{} \citep[voir][p.~2]{Nikonovich11} donne une date
d'enregistrement en~1950-1951, mais cela est contredit par \FMalik{}
\citep[voir][p.~66]{Malik} pour les Préludes \Opus{28}.

\subsection{Melodija D011373/4}

\PublishDate{1963}
\Chopin{} (Nocturne \Opus{37} \Number{2}, Dix Mazurkas \Opus{67} \Number{4},
\Opus{6} \Number{3}, \Opus{41} \Number{2}, \Opus{17} \Number{1}, \Opus{33}
\Number{1}, \Opus{6} \Number{4}, \Opus{63} \Number{1}, \Opus{63} \Number{2},
\Opus{68} \Number{3} et \Opus{68} \Number{1}, Deux Valses \Opus{70}
\Number{2} et \Opus{64} \Number{1}, Nocturne \Opus{9} \Number{2}, Étude
\Opus{25} \Number{3}, Nocturne \Opus{48} \Number{1}, Valse \Opus{64}
\Number{3}, Mazurka \Opus{33} \Number{3}).
Enregistrements 1949-11-21.

\subsection{Melodija D011441/2}

\PublishDate{1963}
\Liszt{} (Sonate S~178 [1960-10-11]).
\Schumann{} (Études symphoniques \Opus{13} et \Opus{posthume} [1959-11-18]).

\subsection{Melodija D00011585/6}

\PublishDate{1963}
\Chopin{} (Deux Valses \Opus{64} \Number{1} et \Opus{64} \Number{3}, Étude
\Opus{25} \Number{3})~: voir Melodija D011373/4.
\Rachmaninov{} (Deux Préludes \Opus{3} \Number{2} et \Opus{32}
\Number{12})~: voir Melodija D09405/6.

\subsection{Melodija D013379/80}

\PublishDate{1964}
\Borodine{} (Petite Suite).
\Liadov{} (Trois Préludes \Opus{46} \Number{1}, \Opus{39} \Number{2} et
\Opus{36} \Number{3}).
\Rachmaninov{} (Prélude \Opus{23} \Number{4}).
\Scriabine{} (Deux Préludes \Opus{11} \Number{13} et \Opus{11} \Number{14}).
\Prokofiev{} (Cinq Visions fugitives \Opus{22} \Number{1}, \Opus{22}
\Number{2}, \Opus{22} \Number{3}, \Opus{22} \Number{7} et \Opus{22}
\Number{11}).
\Kabalevski{} (Sonatine \Opus{13} \Number{1}).
\Goltz{} (Scherzo en \kE mineur).
\Mendelssohn{} (Étude \Opus{104} \Number{3}).
\Debussy{} (Prélude General Lavine -- eccentric L~123 \Number{VI}).
Enregistrements 1954-06-11.

\subsection{Melodija D013459/60}

\PublishDate{1964}
\Mozart{} (Fantaisie K~475).
\Schubert{} (Deux Impromptus D~899 \Number{3} et D~899 \Number{4}).
\Chopin{} (Deux Nocturnes \Opus{15} \Number{2} et \Opus{15} \Number{1}, Deux
Scherzos \Opus{20} et \Opus{31}).
Enregistrements 1960-05-13.

\subsection{Melodija D013539/40}

\PublishDate{1964}
\Schubert{} (Sonate D~960 [1960-10-14]).
\Schubert{}/\Liszt{} (Der Doppelgänger [1960-10-11], Erlkönig [1960-10-11]).

\subsection{Melodija D013567/8}

\PublishDate{1964}
\Schumann{} (Fantaisie \Opus{17}, Carnaval \Opus{9}).
Enregistrements 1959-11-18.

\subsection{Melodija D013699/700}

\PublishDate{1964}
\Chopin{} (Polonaise \Opus{26} \Number{1}, Valse \Opus{69} \Number{1},
Quatre Mazurkas \Opus{63} \Number{2}, \Opus{30} \Number{3}, \Opus{30}
\Number{4} et \Opus{50} \Number{3})~: enregistrements en~1960.
\Schubert{}/\Liszt{} (Litanei)~: dernier enregistrement en studio
1960-12-11.
\Liszt{} (Sposalizio S~161 \Number{1})~: dernier enregistrement en studio
1960-12-11.
\Scriabine{} (Étude \Opus{8} \Number{8} [1959-09-04], Sonate \Opus{19}
[premier mouvement seul, dernier enregistrement en studio 1960-12-11]).
\INikonovich{} \citep[voir][p.~3]{Nikonovich11} donne une date
d'enregistrement en~1959 pour le premier mouvement de la Sonate \Opus{19},
mais cela est contredit par \FMalik{} \citep[voir][p.~76]{Malik}.

\subsection{Melodija D014845/6}

\PublishDate{1964}
\Mozart{} (Fantaisie K~396 [1952]).
\Schumann{} (Quasi Variazioni~: Andantino de \CWieck{} \Opus{14~(III)}
[1953], Kreisleriana \Opus{16} [1952]).

\subsection{Melodija D014915/6}

\PublishDate{1964}
\Liadov{} (Quatre Préludes \Opus{57} \Number{1} [1949], \Opus{31} \Number{2}
[1949], \Opus{40} \Number{3} [1949] et \Opus{36} \Number{3} [1949],
Novelette \Opus{20} [1951]).
\Glazounov{} (Prélude \Opus{49} \Number{1} [1952], Prélude et fugue
\Opus{101} \Number{1} [1952]).
\Scriabine{} (Deux Mazurkas \Opus{3} \Number{6} [1952] et \Opus{3}
\Number{9} [1952], Prélude \Opus{13} \Number{6} [1946], Trois Préludes
\Opus{16} \Number{1} [1948], \Opus{16} \Number{3} [1952] et \Opus{16}
\Number{4} [1952], Prélude \Opus{22} \Number{1} [1946], Deux Études \Opus{8}
\Number{7} [1948] et \Opus{8} \Number{9} [1946], Valse \Opus{38} [1946]).

\subsection{Melodija D014987/8}

\PublishDate{1965}
\Schubert{} (Sonate D~784).
\Schubert{}/\Liszt{} (Die Stadt, Die junge Nonne).
\Schubert{} (Fantaisie \Quote{Wanderer} D~760).
\Schubert{}/\Liszt{} (Am Meer, Auf dem Wasser zu singen).
Enregistrements 1953-12-25.

\subsection{Melodija D014989/90}

\PublishDate{1965}
\Scriabine{} (Huit Préludes \Opus{11} \Number{4}, \Opus{16} \Number{5},
\Opus{9} \Number{1}, \Opus{13} \Number{6}, \Opus{11} \Number{17}, \Opus{11}
\Number{19}, \Opus{11} \Number{20} et \Opus{11} \Number{23} [1960-01-08],
Mazurka \Opus{40} \Number{2} [1960-02-02], Valse \Opus{38}, Sonate
\Opus{30}, Poème \Opus{34} [1960-05-13], Cinq Poèmes \Opus{63} \Number{1},
\Opus{69} \Number{1}, \Opus{69} \Number{2}, \Opus{71} \Number{1} et
\Opus{71} \Number{2} [1960-01-08], Deux Danses \Opus{73} \Number{1} et
\Opus{73} \Number{2} [1960-01-08], Trois Préludes \Opus{74} \Number{1},
\Opus{74} \Number{3} et \Opus{74} \Number{4} [1958-06-08], Sonate \Opus{66}
[1958-06-08]).

\subsection{Melodija D015001/2}

\PublishDate{1965}
\Beethoven{} (Sonate \Opus{27} \Number{2} [1952-02-03]).
\Chopin{} (Six Préludes \Opus{28} \Number{13}, \Opus{28} \Number{14},
\Opus{28} \Number{15}, \Opus{28} \Number{17}, \Opus{28} \Number{20} et
\Opus{28} \Number{22} [1949-11-21], Fantaisie \Opus{49} [1949-10-20],
Nouvelle Étude en \kA \Flat majeur [1949-10-20], Polonaise \Opus{44}
[1948-01-05]).

\subsection{Melodija D015025/6}

\PublishDate{1965}
\Beethoven{} (Andante favori WoO~57 [1960-10-22/28]).
\Chopin{} (Nocturne \Opus{15} \Number{2} [1960-10-22/28], Deux Études
\Opus{10} \Number{3} et \Opus{10} \Number{4} [1960-10-22/28], Ballade
\Opus{47} [1960-10-22/28]).
\Liszt{} (Méphisto-valse \Number{2} S~515 [1960-10-22/28]).
\Rachmaninov{} (Deux Préludes \Opus{32} \Number{5} et \Opus{32} \Number{12}
[1960-10-28]).
\Scriabine{} (Quatre Études \Opus{8} \Number{11} [1960-05-13], \Opus{42}
\Number{3} [1958-06-08], \Opus{42} \Number{4} [1958-06-08] et \Opus{42}
\Number{5} [1960-02-02]).

\subsection{Melodija D016161/2}

\PublishDate{1965}
\Borodine{} (Petite Suite [1950]).
\Kabalevski{} (Sonatine \Opus{13} \Number{1} [1953]).
\Prokofiev{} (Contes de la vieille grand-mère \Opus{31} [1946], Vision
fugitive \Opus{22} \Number{7} [1946], Sarcasme \Opus{17} \Number{3} [1946],
Six Pièces \Opus{12} \Number{2}, \Opus{12} \Number{3}, \Opus{12} \Number{6},
\Opus{12} \Number{7}, \Opus{12} \Number{8} et \Opus{12} \Number{9} [1953]).
\INikonovich{} \citep[voir][p.~4]{Nikonovich11} donne une date
d'enregistrement en~1952 pour la Sonatine \Opus{13} \Number{1} de
\Kabalevski{}, mais cela est contredit par \FMalik{}
\citep[voir][p.~66]{Malik}.

\subsection{Melodija D016163/4}

\PublishDate{1965}
\Beethoven{} (Sonate \Opus{111} [1951]).
\Scriabine{} (Sonate \Opus{23} [1946]).

\subsection{Melodija D016179/80}

\PublishDate{1965}
\Chopin{} (Ballade \Opus{47} [1951], Deux Nocturnes \Opus{15} \Number{1}
[1948] et \Opus{27} \Number{2} [1952], Prélude \Opus{28} \Number{21} [1950],
Polonaise \Opus{53} [1947]).
\Liszt{} (Sonetto~123 del Petrarca S~161 \Number{6} [1952], Églogue S~160
\Number{7} [1952], Sposalizio S~161 \Number{1} [1949], Il penseroso S~161
\Number{2} [1949], Canzonetta del Salvator Rosa S~161 \Number{3} [1949]).

\subsection{Melodija D016257/8}

\PublishDate{1965}
\Debussy{} (Prélude Canope L~123 \Number{X} [1952-01-28], Prélude Feuilles
mortes L~123 \Number{II} [1952-01-28], Serenade of the Doll L~113
\Number{III} [1950-1952]).
\Liadov{} (Une Tabatière à musique \Opus{32} [1950-1952], Barcarolle
\Opus{44} [1950-1952]).
\Blumenfeld{} (Deux Fragments lyriques \Opus{47} [1950-1952]).
\Scriabine{} (Trois Études \Opus{8} \Number{7} [1950-1952], \Opus{8}
\Number{10} [1950-1952] et \Opus{8} \Number{12} [1950-1952], Sonate
\Opus{30} [1954-06-11], Deux Préludes \Opus{67} \Number{1} et \Opus{67}
\Number{2} [1955-01-14], Sonate \Opus{53} [1950-1952]).
\Prokofiev{} (Sarcasme \Opus{17} \Number{5} [1950-1952]).

\subsection{Melodija D00016611/2}

\PublishDate{1965}
\Scriabine{} (Deux Poèmes \Opus{32} \Number{1} et \Opus{32} \Number{2}~:
voir Melodija D08779/80).
\Scriabine{} (Morceau \Opus{45} \Number{1} [1960-01-08], Morceau \Opus{59}
\Number{1} [1960-01-08]).
\Scriabine{} (Poème \Opus{72}~: voir Melodija D04888/9).

\subsection{Melodija D016907/8}

\PublishDate{1966}
\Scriabine{} (Vingt-quatre Préludes \Opus{11} [1950-1960], Quatre Préludes
\Opus{17} \Number{1} [1952], \Opus{17} \Number{3} [1948], \Opus{17}
\Number{4} [1952] et \Opus{17} \Number{6} [1952], Prélude \Opus{27}
\Number{1} [1946], Trois Préludes \Opus{33} \Number{1} [1953], \Opus{33}
\Number{2} [1953] et \Opus{33} \Number{3} [1953], Prélude \Opus{48}
\Number{2} [1951], Trois Préludes \Opus{39} \Number{2} [1951], \Opus{39}
\Number{3} [1953] et \Opus{39} \Number{4} [1948], Quatre Études \Opus{42}
\Number{2} [1950], \Opus{42} \Number{3} [1959], \Opus{42} \Number{4} [1948]
et \Opus{42} \Number{6} [1950]).
\INikonovich{} \citep[voir][p.~4]{Nikonovich11} donne les dates
d'enregistrement entre~1951 et~1956 pour les Préludes \Opus{11}, mais cela
est contredit par \FMalik{} \citep[voir][p.~72]{Malik}.

\subsection{Melodija D019187/8}

\PublishDate{1967}
\Schubert{}/\Liszt{} (Der Müller und der Bach [1948-01-05], Frühlingsglaube
[1953-12-25], Der Doppelgänger [1948-01-05], Die Stadt [1953-12-25], Am Meer
[1953-12-25]).
\Schumann{} (Des Abends \Opus{12} \Number{1} [1959-11-18], Arabesque
\Opus{18} [1959-11-18], Sonate \Opus{11} [1960-05-13]).

\subsection{Melodija D19493/4}

\PublishDate{1967}
\Liadov{} (Prélude \Opus{11} \Number{1}, Une Tabatière à musique \Opus{32},
Valse \Opus{57} \Number{2}, Mazurka \Opus{57} \Number{3}).
\Medtner{} (Conte de fées (Skazka) \Opus{20} \Number{2}).
\Scriabine{} (Deux Préludes \Opus{27} \Number{1} et \Opus{27} \Number{2}).
\Chopin{} (Mazurka \Opus{41} \Number{2}, Valse \Opus{70} \Number{1}, Étude
\Opus{10} \Number{4}).
\Liszt{} (Gnomenreigen S~145 \Number{2}).
\Goltz{} (Prélude en \kE mineur, Scherzo en \kE mineur).
Enregistrements entre~1937 et~1947.

\subsection{Melodija D019637/8, \Quote{En direct au musée \Scriabine{},
\Volume{1}}}

\PublishDate{1967}
\Schubert{}/\Liszt{} (Soirée de Vienne \Number{7} S~427 \Number{7}).
\Schumann{} (Quasi Variazioni~: Andantino de \CWieck{} \Opus{14~(III)}).
\Chopin{} (Nocturne \Opus{15} \Number{3}, Valse \Opus{69} \Number{2}).
\Liszt{} (Fantaisie et fugue sur le thème BACH S~529, Nuages gris S~199).
\Debussy{} (Reflets dans l'eau L~110 \Number{I}, Prélude de la Suite pour le
piano L~95 \Number{I}, Doctor Gradus ad Parnassum L~113 \Number{I}, Serenade
of the doll L~113 \Number{III}, Prélude Minstrels L~117 \Number{XII}).
Enregistrements entre~1954 et~1961.

\subsection{Melodija D019639/40, \Quote{En direct au musée \Scriabine{},
\Volume{2}}}

\PublishDate{1967}
\Scriabine{} (Sonate \Opus{6} [quatrième mouvement seul], Huit Préludes
\Opus{35} \Number{1}, \Opus{35} \Number{2}, \Opus{37} \Number{1}, \Opus{37}
\Number{2}, \Opus{37} \Number{3}, \Opus{37} \Number{4}, \Opus{27}
\Number{1} et \Opus{27} \Number{2}, Sonate \Opus{19}, Mazurka \Opus{40}
\Number{1}, Quasi-valse \Opus{47}, Deux Préludes \Opus{56} \Number{1} et
\Opus{45} \Number{3}, Deux Morceaux \Opus{56} \Number{2} et \Opus{56}
\Number{3}, Morceau \Opus{45} \Number{2}, Trois Morceaux \Opus{49}
\Number{1}, \Opus{49} \Number{2} et \Opus{49} \Number{3}, Deux Morceaux
\Opus{57} \Number{1} et \Opus{57} \Number{2}, Morceau \Opus{51} \Number{4},
Morceau \Opus{52} \Number{2}, Poème \Opus{63} \Number{1}, Sonate \Opus{62}).
Enregistrements entre~1954 et~1961.

\subsection{Melodija D019641/2, \Quote{En direct au musée \Scriabine{},
\Volume{3}}}

\PublishDate{1967}
\Scriabine{} (Poème \Opus{41}, Dix Préludes \Opus{33} \Number{1}, \Opus{31}
\Number{2}, \Opus{33} \Number{2}, \Opus{39} \Number{2}, \Opus{39}
\Number{3}, \Opus{48} \Number{1}, \Opus{48} \Number{2}, \Opus{39}
\Number{4}, \Opus{31} \Number{4} et \Opus{31} \Number{3}, Poème-nocturne
\Opus{61}, Trois Études \Opus{65} \Number{1}, \Opus{65} \Number{2} et
\Opus{65} \Number{3}, Cinq Préludes \Opus{74} \Number{1}, \Opus{74}
\Number{2}, \Opus{74} \Number{3}, \Opus{74} \Number{4} et \Opus{74}
\Number{5}).
\Chostakovitch{} (Deux Préludes et fugues \Opus{87} \Number{9} et \Opus{87}
\Number{3}).
\Prokofiev{} (Dix Visions fugitives \Opus{22} \Number{1}, \Opus{22}
\Number{2}, \Opus{22} \Number{3}, \Opus{22} \Number{4}, \Opus{22}
\Number{6}, \Opus{22} \Number{7}, \Opus{22} \Number{10}, \Opus{22}
\Number{12}, \Opus{22} \Number{17} et \Opus{22} \Number{18}, Cinq Sarcasmes
\Opus{17} \Number{1}, \Opus{17} \Number{2}, \Opus{17} \Number{3}, \Opus{17}
\Number{4} et \Opus{17} \Number{5}).
Enregistrements entre~1954 et~1961.

\subsection{Melodija D019847/8}

\PublishDate{1967}
\Mozart{} (Fantaisie K~475 [1952]).
\Chopin{} (Nocturne \Opus{9} \Number{2} [1950], Trois Valses \Opus{64}
\Number{1} [1947], \Opus{64} \Number{3} [1946] et \Opus{70} \Number{1}
[1946], Mazurka \Opus{41} \Number{1} [1960]).
\Schumann{} (Carnaval \Opus{9} [1951]).
Selon \citet{Malik} et \citet{Masuda}, le numéro de catalogue est Melodija
D019847/8~; selon \citet{Nikonovich11}, \citet{Rossi} et \citet{White}, il
s'agit du numéro de catalogue Melodija D019841/2.

\subsection{Melodija 33D~29437/8}

\PublishDate{1970}
\Borodine{} (Au couvent, extrait de la Petite Suite).
Ce disque n'est pas mentionné par \INikonovich{}
\citep[voir][]{Nikonovich11} et seule une partie de sa face~B (29438) est
consacrée à \VSofronitsky{}.
Voir \citet{Recordssu}.

\subsection{Krugozor~1970 \Number{6} -- GD~0002011/20, GD~0002023/24}

\PublishDate{1970}
\Scriabine{} (Prélude \Opus{11} \Number{4}).
Ce disque n'est pas mentionné par \INikonovich{}
\citep[voir][]{Nikonovich11} et seul le disque 33GD~0002018 est consacré à
\VSofronitsky{}~; il s'agit à la fois d'une leçon de \VSofronitsky{} et d'un
extrait de récital.
Voir \citet{Recordssu}.

\subsection{Melodija C50~05305/6}

\PublishDate{1974}
\Kabalevski{} (Sonatine \Opus{13} \Number{1}).
Ce disque n'est pas mentionné par \INikonovich{}
\citep[voir][]{Nikonovich11} et seule une partie du disque est consacrée à
\VSofronitsky{}.
Voir \citet{Recordssu}.

\subsection{Melodija C10~11625/6}

\PublishDate{1979}
\Prokofiev{} (Sarcasmes \Opus{17} et Contes de la vieille grand-mère
\Opus{31}).
Ce disque n'est pas mentionné par \INikonovich{}
\citep[voir][]{Nikonovich11} et seule une partie du disque est consacrée à
\VSofronitsky{}.
Voir \citet{Recordssu}.

\subsection{Krugozor~1983 \Number{8} -- G~92-10145/56}

\PublishDate{1983}
\Bach{}/\Ziloti{} (Arrangement pour piano en \kB mineur du Prélude
\Number{10} en \kE mineur BWV~855a de \Bach{} par \Ziloti{}).
\Liadov{} (\emph{A Musical Snuffbox} \Opus{32}).
Ce disque n'est pas mentionné par \INikonovich{}
\citep[voir][]{Nikonovich11} et seul le disque G~92-10149 est consacré à
\VSofronitsky{}.
Voir \citet{Recordssu}.

\subsection{Melodija M90~48393/404}

\PublishDate{1988}
\Chopin{} (Nocturne \Opus{9} \Number{2}).
Ce disque n'est pas mentionné par \INikonovich{}
\citep[voir][]{Nikonovich11} et seule une partie de la face~A de son
cinquième disque (48401) est consacrée à \VSofronitsky{}.
Voir \citet{Recordssu}.

\subsection{Melodija C10~26297~006}

\PublishDate{1988}
\Scriabine{} (Vers la flamme \Opus{72}).
Ce disque n'est pas mentionné par \INikonovich{}
\citep[voir][]{Nikonovich11} et seule une partie du disque est consacrée à
\VSofronitsky{}.
Voir \citet{Recordssu}.

\subsection{Melodija C50~26509~007}

\PublishDate{1988}
\Chopin{} (Valse non mentionnée).
Ce disque n'est pas mentionné par \INikonovich{}
\citep[voir][]{Nikonovich11} et seule une partie du disque est consacrée à
\VSofronitsky{}.
Voir \citet{Recordssu}.

\subsection{Melodija M10~49011/2~001}

\PublishDate{1990}
\Beethoven{} (\emph{Adagio sostenuto}, extrait de la Sonate \Opus{27}
\Number{2}).
Ce disque n'est pas mentionné par \INikonovich{}
\citep[voir][]{Nikonovich11} et seule une partie de sa face~B (49012) est
consacrée à \VSofronitsky{}.
Voir \citet{Recordssu}.

\subsection{Melodija R10~00783/6}

\PublishDate{1992}
\Scriabine{} (Étude \Opus{42} \Number{4} [1948]).
Ce disque n'est pas mentionné par \INikonovich{}
\citep[voir][]{Nikonovich11} et seule une partie de la face~A de son premier
disque (00783) est consacrée à \VSofronitsky{}.
Voir \citet{Recordssu}.

\subsection{Artia (Melodija) MK-1562}

\Scriabine{} (Sonate \Opus{23}, Sonate \Opus{68}, Deux Danses \Opus{73}
\Number{2} et \Opus{73} \Number{1}, Poème \Opus{72}).
Exporté par Melodija en~1961, sans doute transcrit à partir de Melodija
D04888/9 \citep[voir][p.~94]{White}.

\subsection{Artia (Melodija) MK-D08779/80}

Voir Melodija D08779/80.

\subsection{Artia (Melodija) MK-D013700}

\Scriabine{} (Sonate \Opus{19} [premier mouvement seul], Étude \Opus{8}
\Number{8}).
Ce disque peut contenir des œuvres d'autres compositeurs \citep[voir][p.~94,
note~9]{White}.

\subsection{Artia (Melodija) MK-D091553}

\Scriabine{} (Trois Morceaux \Opus{2} \Number{1}, \Opus{2} \Number{2} et
\Opus{2} \Number{3}, Polonaise \Opus{21}, Poème \Opus{44}, Mazurkas,
Préludes, Impromptus).
Ce disque peut contenir des œuvres d'autres compositeurs \citep[voir][p.~94,
note~10]{White}.

\subsection{Melodija Russian Disc 10~1071/4}

\PublishDate{1992}
\Scriabine{} (Deux Danses \Opus{73} \Number{1} [1960-07-12] et \Opus{73}
\Number{2} [1960-07-12], Poème \Opus{63} \Number{1} [1960-07-12], Prélude
\Opus{67} \Number{1} [1960-07-12], Sonate \Opus{66} [1958-05-05]).

\subsection{Melodija/Eurodisc 85993~XFK}

Les maisons d'édition Melodija et Eurodisc ont en outre publié le boîtier de
trois disques vinyles Melodija/Eurodisc 85993~XFK (\Schumann{} [Kreisleriana
\Opus{16}, Quasi Variazioni~: Andantino de \CWieck{} \Opus{14~(III)},
Arabesque \Opus{18}], \Chopin{} [Polonaise \Opus{26} \Number{1}, Nocturnes
\Opus{27} \Number{1} et \Opus{27} \Number{2}, Études \Opus{10} \Number{3} et
\Opus{10} \Number{4}, Mazurka \Opus{50} \Number{3}, Impromptu \Opus{51},
Valses \Opus{70} \Number{2} et \Opus{70} \Number{3}, Ballade \Opus{47}],
\Scriabine{} [Étude \Opus{8} \Number{11}, Sonate \Opus{23}, Sonate
\Opus{68}, Poème \Quote{Vers la flamme} \Opus{72}, Danse \Quote{Guirlandes}
\Opus{73} \Number{1}, Danse \Quote{Flammes sombres} \Opus{73} \Number{2}]).

\subsection{Melodija/JVC~Victor VIC~4516}

\PublishDate{1982}
Les maisons d'édition Melodija et JVC~Victor ont en outre publié le disque
vinyle japonais Melodija/JVC~Victor VIC~4516 (\Liszt{} [Sonate en \kB mineur
S~178, Sonetto~104 del Petrarca S~161 \Number{5} et Gnomenreigen S~145
\Number{2}], \Rachmaninov{} [Prélude \Opus{32} \Number{5} et Prélude
\Opus{32} \Number{12}], \Chopin{} [Nocturne \Opus{15} \Number{2}] et
\Scriabine{} [Poème \Opus{32} \Number{1}]).

\section{Autres parutions sur disque vinyle}

\subsection{Le Chant du monde LDX~78764/5}

\PublishDate{1985}
La maison d'édition Le Chant du monde a publié le double disque vinyle
LDX~78764/5 consacré à \Scriabine{} (Sonate \Opus{19} [premier mouvement
seul], Sonate \Opus{23}, Sonate \Opus{30}, Sonate \Opus{53}, Sonate
\Opus{66}, Sonate \Opus{68}, Sonate \Opus{70} et enregistrement intégral des
Études \Opus{8}).
Les mêmes versions des Sonates, sauf l'\Opus{19} datée 1960-12-11, ont été
reprises sur le disque compact Le Chant du monde LDC~278764~; les mêmes
versions des Études ont été reprises sur le disque compact Le Chant du monde
LDC~278765, en couplage avec un enregistrement intégral entre~1950 et~1960
des Préludes \Opus{11}.

\subsection{Harmonia Mundi HMC~5169}

\PublishDate{1986}
La maison d'édition Harmonia Mundi a publié le disque vinyle HMC~5169
consacré à \Schubert{} (Sonate D~960) et à cinq lieder de \Schubert{}
transcrits par \Liszt{} (Aufenthalt, Der Doppelgänger, Erlkönig,
Frühlingsglaube et Der Müller und der Bach).
Les mêmes versions de ces œuvres ont été reprises sur le disque compact
Harmonia Mundi HMX~1905169.

\section{Numérisation de disques vinyles}

Du~26~août~2005 au~6~avril~2010, \KMusatov{} \citep[voir][]{Musatov} a
effectué la numérisation d'un très grand nombre de documents d'archives de
\VSofronitsky{} dans leurs premières éditions sur disque vinyle~; voici la
liste des disques vinyles numérisés -- pour leur contenu détaillé, voir
sections précédentes~: Melodija D013539/40 (\Schubert{},
\Schubert{}/\Liszt{}), Melodija D014845/6 (\Mozart{}, \Schumann{}), Melodija
D014989/90 (\Scriabine{}), Melodija D010315/6 (\Schubert{}/\Liszt{},
\Liszt{}), Melodija D011373/4 (\Chopin{}), Melodija D016257/8 (\Debussy{},
\Liadov{}, \Blumenfeld{}, \Scriabine{}, \Prokofiev{}), Melodija D019641/2
(\Scriabine{}, \Chostakovitch{}, \Prokofiev{}), Melodija D09155/6
(\Scriabine{}), Melodija D09405/6 (\Rachmaninov{}), Melodija D011441/2
(\Liszt{}, \Schumann{}), Melodija D013699/700 (\Chopin{},
\Schubert{}/\Liszt{}, \Liszt{}, \Scriabine{}), Melodija D016907/8
(\Scriabine{}), Melodija D04888/9 (\Scriabine{}), Melodija D08723/4
(\Chopin{}), Melodija D08725/6 (\Schubert{}), Melodija D013379/80
(\Borodine{}, \Liadov{}, \Rachmaninov{}, \Scriabine{}, \Prokofiev{},
\Kabalevski{}, \Goltz{}, \Mendelssohn{}, \Debussy{}), Melodija D08763/4
(\Schumann{}), Melodija D08779/80 (\Scriabine{}), Melodija D09085/6
(\Chopin{}), Melodija D09145/6 (\Beethoven{}, \Mendelssohn{}).

\section{Parutions sur disque compact}

Un tableau en page~\pageref{tab:CDData} reprend les données de publication
des disques compacts consacrés à \VSofronitsky{} -- éditeur et numéro de
catalogue~; code~UPC, code~EAN ou numéro~ISBN~; date de publication.
Les notes appelées par des astérisques sont regroupées en fin de tableau.

\subsection{Série de disques compacts éditée par Arlecchino}

La série éditée par la firme \emph{Arlecchino} comporte vingt disques
compacts (\Volume{1}~: ARL~1 [\Schumann{}]~; \Volume{2}~: ARL~2
[\Schubert{}]~; \Volume{3}~: ARL~12 [\Schumann{}]~; \Volume{4}~: ARL~28
[\Liszt{}]~; \Volume{5}~: ARL~31 [\Scriabine{}~I]~; \Volume{6}~: ARL~39
[\Schumann{}]~; \Volume{7}~: ARL~40 [\Beethoven{}]~; \Volume{8}~: ARL~42
[\Scriabine{}~II]~; \Volume{9}~: ARL~41 [\Chopin{}]~; \Volume{10}~: ARL~61
[\Chopin{}]~; \Volume{11}~: ARL~62 [\Scriabine{}~III]~; \Volume{12}~: ARL~67
[\Schubert{}]~; \Volume{13}~: ARL~95 [\Chopin{}]~; \Volume{14}~: ARL~119
[\Scriabine{}~IV]~; \Volume{15}~: ARL~132 [\Rachmaninov{}]~; \Volume{16}~:
ARL~155 [\Schumann{}]~; \Volume{17}~: ARL~174 [\Debussy{}, \Borodine{},
\Liadov{}, \Medtner{}, \Blumenfeld{}, \Goltz{}, \Glazounov{}]~;
\Volume{18}~: ARL~183 [\Schubert{}]~; \Volume{19}~: ARL~188
[\Scriabine{}~V]~; \Volume{20}~: ARL~A11 [\Mozart{}, \Beethoven{},
\Mendelssohn{}]).
Dans cette série, les notes, et en particulier les attributions des dates
des enregistrements, sont soit inexistantes, soit inexactes, soit
incomplètes \citep[voir][p.~62]{Juban}~; en réalité, une partie du travail
discographique réalisé par \FMalik{} \citep[voir][]{Malik} a consisté à
rectifier ces attributions, au terme d'écoutes comparatives systématiques.
En ce qui concerne la qualité technique de la série Arlecchino, les avis
divergent~: la plupart des critiques estiment que ces transferts sont
déplorables, et insistent sur le fait qu'il s'agit de transferts à partir
des disques vinyles de Melodija et non d'un travail d'ingénierie à partir
des bandes d'origine \citep[voir][p.~62]{Juban}, mais certains critiques
soulignent que plusieurs transferts de la série sont en réalité de bonne
qualité, et insistent sur le fait qu'il n'y a aucune raison pour qu'un
disque vinyle vieillisse mal s'il a été entretenu avec soin, tandis que les
bandes d'origine sont soumises au vieillissement.
En revanche, la qualité éditoriale de la série Arlecchino fait, comme on l'a
indiqué plus haut, l'objet de critiques unanimes~: attributions erronées
voire fallacieuses, erreurs présentes même dans l'identification des œuvres
enregistrées,~etc.
Le \Volume{18} (ARL~183) est bien consacré dans son entièreté à \Schubert{},
et non à \Mozart{} comme indiqué par erreur sur le disque compact lui-même
ainsi que sur la tranche de son boîtier en plastique.
Dans le même ordre d'idées, on ne dispose d'aucun enregistrement des
Impromptus sur une Romance de \CWieck{}, \Opus{5}, de \Schumann{} par
\VSofronitsky{}~; il semble même que l'œuvre ne figurait pas au répertoire
du pianiste \citep[voir][p.~75]{White}.
Le disque compact Arlecchino ARL~1 (\Volume{1}) confond cette œuvre avec le
troisième mouvement de la Sonate, \Opus{14} (\Quote{Quasi Variazioni~:
Andantino de \CWieck{}}).
Quoi qu'il en soit, la série éditée par Arlecchino ne peut satisfaire les
collectionneurs de disques en raison de sa présentation éditoriale et de la
qualité technique très variable de ses transferts
\citep[voir][p.~62]{Juban}.

\subsection{Série de disques compacts éditée par Denon}

La série éditée par la firme \emph{Denon} \citep[voir][]{Denon03, Denon05,
Denon06, Graham} comporte quinze volumes, dix-neuf disques compacts~; la
série d'origine a été rééditée en grande partie (\Volume{1}~: COCO-80074/5~=
COCQ-83673/4~; \Volume{2}~: COCO-80149/50~= COCQ-83968/9~; \Volume{3}~:
COCO-80187~= COCQ-83970~; \Volume{4}~: COCO-80188, non réédité~;
\Volume{5}~: COCO-80189, non réédité~; \Volume{6}~: COCO-80383/4~=
COCQ-83667/8~; \Volume{7}~: COCO-80385/6~= COCQ-83669/70~; \Volume{8}~:
COCO-80568~= COCQ-83671~; \Volume{9}~: COCO-80569~= COCQ-84241~;
\Volume{10}~: COCO-80714, non réédité~; \Volume{11}~: COCO-80715~=
COCQ-83973~; \Volume{12}~: COCO-80815~= COCQ-83672~; \Volume{13}~:
COCO-80816~= COCQ-84242~; \Volume{14}~: COCQ-83286~= COCQ-83971~;
\Volume{15}~: COCQ-83490~= COCQ-83972).
Seuls les deux derniers volumes de la série -- volume~14 et volume~15 --
sont postérieurs à la publication dans~IPQ de la discographie établie par
\FMalik{} \citep[voir][]{Malik}.
Sur la qualité technique de la série Denon, les avis sont très homogènes~:
il s'agit d'un travail d'ingénierie minutieux, très souvent à partir des
bandes d'origine, et le résultat est un son dynamique et clair qui préserve
le timbre de l'instrument, et l'acoustique de la salle dans le cas des
enregistrements en public \citep[voir][p.~62]{Juban}.
Le contenu éditorial de la série Denon est accessible en japonais et dans
cette seule langue, ce que l'on ne peut que regretter.
Dans tous les cas, la table des matières en anglais est cependant donnée.
La série éditée par Denon a été publiée sous licence de la Radio Ostankino
de Moskva \citep[voir][p.~11]{Nikonovich11}.
Trois ou quatre disques compacts manquent, dans cette série, pour compléter
la publication de tous les enregistrements conservés à la Fondation, mais
tout ce matériau manquant, de même que des enregistrements en provenance
d'autres archives, est maintenant publié par la maison d'édition russe Vista
Vera \citep[voir][p.~11]{Nikonovich11}.

\subsection{Série de disques compacts éditée par Vista Vera}

La série éditée par la firme \emph{Vista Vera} \citep[voir][]{VistaVera} a
été initiée en~1997 (\Volume{1}~: VVCD-00014~; \Volume{2}~: VVCD-00024~;
\Volume{3}~: VVCD-00031~; \Volume{4}~: VVCD-00091~; \Volume{5}~:
VVCD-00093~; \Volume{6}~: VVCD-00113~; \Volume{7}~: VVCD-00118-2 [deux
disques]~; \Volume{8}~: VVCD-00125~; \Volume{9}~: VVCD-00136~; \Volume{10}~:
VVCD-00137~; \Volume{11}~: VVCD-00148-2 [deux disques]~; \Volume{12}~:
VVCD-00155~; \Volume{13}~: VVCD-00164~; \Volume{14}~: VVCD-00182~;
\Volume{15}~: VVCD-00198~; \Volume{16}~: VVCD-00203~; \Volume{17}~:
VVCD-00204~; \Volume{18}~: VVCD-00218~; \Volume{19}~: VVCD-00241).
En~2009, une série supplémentaire a été initiée~; elle est consacrée aux
enregistrements effectués au musée \Scriabine{} (\Volume{1}~: VVCD-00195~;
\Volume{2}~: VVCD-00210~; \Volume{3}~: VVCD-00213~; \Volume{4}~:
VVCD-00222~; \Volume{5}~: VVCD-00223~; \Volume{6}~: VVCD-00224~;
\Volume{7}~: VVCD-00225~; \Volume{8}~: VVCD-00233~; \Volume{9}~:
VVCD-00248~; \Volume{10}~: VVCD-00249).
Presque tous ces disques sont postérieurs à la discographie due à
\FMalik{} \citep[voir][]{Malik}.
En ce qui concerne la qualité technique de la série Vista Vera, l'état des
lieux est assez contrasté.
Les volumes~1 à~5 -- à savoir \Scriabine{}, \Schumann{}, \Schubert{},
\Rachmaninov{} et \Scriabine{} -- sont excellents à tous les points de vue.
Bien qu'ils utilisent presque toujours les bandes d'origine, la plupart des
volumes suivants s'attachent à supprimer coûte que coûte le souffle présent
-- ce qui assourdit dans une large mesure la perspective sonore (instrument
et salle).
Dans les cas les plus extrêmes, il ne reste plus guère de signal au-dessus
de~8\,kHz ou~9\,kHz.
Cependant, la qualité de cette série reste homogène, et certains volumes
ultérieurs sont bons voire très bons -- le volume~8 (\Beethoven{},
\Schubert{}, \Chopin{}), le volume~16 (\Schubert{}, \Schumann{}, \Chopin{}),
le volume~18 (\Borodine{}, \Liadov{}, \Rachmaninov{}, \Scriabine{},
\Prokofiev{}, \Kabalevski{}, \Goltz{}, \Mendelssohn{}, \Debussy{},
\Chopin{}), par exemple.

\subsection{Coffret de disques compacts édité par Scribendum}

Vers la fin de l'année~2019, l'éditeur Scribendum a publié un coffret de
trente-quatre disques compacts intitulé \foreignlanguage{english}{\emph{The
Art of Vladimir Sofronitsky}}.
Dans la partie discographique du présent document, chaque disque compact du
coffret est identifié grâce au numéro de catalogue~SC817 du coffret et au
numéro~\pholder{n} du disque, où l'entier~\textit{n} varie de~01 à~34, sous
la forme~SC817/\pholder{n} plus compacte~; cette convention vaut aussi pour
la liste en page~\pageref{chap:Contenu}.

\subsection{Autres parutions sur disque compact}

De nombreuses autres maisons d'édition ont aussi publié des disques compacts
consacrés aux enregistrements de \VSofronitsky{}, à savoir, par ordre
alphabétique~:
\emph{Agora} (AG~236.2)~;
\emph{AML+music}~;
\emph{Andante} (AN~1190)~;
\emph{Arbiter} (ARB~157~; ARB~164)~;
\emph{Arioso} (ARI~700)~;
\emph{Arkadia} (78571)~;
\emph{Brilliant Classics} (BRIL~8975~; BRIL~9013~; BRIL~9014~; BRIL~9241~;
BRIL~94215)~;
\emph{Le Chant du monde} (LDC~278764~; LDC~278765~; LDC~288032)~;
\emph{Classical Recordings Quarterly} (CRQ CD030)~;
\emph{Classical Records} (CR-004~; CR-014)~;
\emph{Classound} (CLAS~001-022~; CLAS~001-023~; CLAS~001-024~;
CLAS~001-025~; CLAS~001-026~; CLAS~2003-008)~;
\emph{Diapason} (DIAP058)~;
\emph{DiwClassics} (DCL-1001~; DCL-1002)~;
\emph{Documents} (600158, CD~9)~;
\emph{Green Door} (GDCS-0030)~;
\emph{Harmonia Mundi} (HMX~1905169)~;
\emph{Marston Records} (MR~\hbox{54001-2}, CD~3)~;
\emph{Meldac} (MECC~26012~; MECC~26016)~;
\emph{Melodija} (MEL CD~74321~25177-2~; MEL CD~10~00747~; MCD~208~;
MEL CD~10~02237~; MEL CD~10~02312~; MEL CD~10~02395 [édition non commerciale
à tirage limité]~; MEL CD~10~02550~; GP15926 [dématérialisé])~;
\emph{Melodija/Bukok} (DE~0176~; DE~0177~; DE~0178~; DE~0180~; DE~0186)~;
\emph{Le Monde du piano} (\Volume{38}, CD~2)~;
\emph{Monopole} (MONO~018)~;
\emph{Moscow State Conservatoire} (SMC CD~0019~; SMC CD~0020~; SMC
CD~0183/0019, CD~1 [réédition en~2016 du disque compact SMC CD~0019]~; SMC
CD~0019/0020 [réédition en~2017 des disques compacts SMC CD~0019 et SMC
CD~0020 sous la forme d'un boîtier])~;
\emph{Multisonic} (\Quote{Russian Treasure}, MT~310181-2)~;
\emph{Musikstrasse} (MC~2108~; MC~2109)~;
\emph{Olympia} (OCD~208)~;
\emph{Originals} (SH~858)~;
\emph{Palladio} (PD~4131)~;
\emph{Philips} (\Quote{Great Pianists of the 20th Century}, \Volume{91},
456~970-2)~;
\emph{The Piano Library} (PL~282)~;
\emph{Profil Hänssler} (DCD PH15007)~;
\emph{Prometheus Editions} (EDITION003~; EDITION003\OneHalf{} [édition à
tirage limité])~;
\emph{Russian Compact Disc} (\Quote{Talents of Russia -- Russian Piano
School}, RCD~16036~; RCD~16288~; RCD~16289)~;
\emph{Russian Disc} (RD CD~15001)~;
\emph{TKM Records TNS} (\Quote{Living Stage}, LS~4035182)~;
\emph{Urania} (SP~4203~; SP~4205~; SP~4211~; SP~4215~; SP~4258~;
URN~22.299)~;
\emph{Venecija} (CDVE~05218~; CDVE~00014).

\setlongtables
{\fontsize{9}{12pt}\selectfont
\phantomsection\label{tab:CDData}
\begin{longtable}[c]{lll}
 \toprule
 \multicolumn{1}{l}{\textbf{Éditeur et numéro de catalogue}}
 & \multicolumn{1}{l}{\textbf{UPC, EAN ou~ISBN}}
 & \multicolumn{1}{l}{\textbf{Date de publication}}
 \\ \midrule
 \endfirsthead
 \midrule
 \multicolumn{1}{l}{\textbf{Éditeur et numéro de catalogue}}
 & \multicolumn{1}{l}{\textbf{UPC, EAN ou~ISBN}}
 & \multicolumn{1}{l}{\textbf{Date de publication}}
 \\ \midrule
 \endhead
 \midrule
 \multicolumn{3}{r}{\textit{Suite du tableau en page suivante}}
 \\ \midrule
 \endfoot
 \midrule
 \multicolumn{3}{l}{\phantom{*}*\space%
 La date de publication du disque n'est pas connue avec certitude.}
 \\
 \multicolumn{3}{l}{**\space%
 Le disque ne mentionne aucun code~UPC, code~EAN ou numéro~ISBN.}
 \\ \bottomrule
 \endlastfoot
 Agora (Musikstrasse) AG~236.2
 & EAN~8018430236021
 & 1999 \\
 AML+music (\Quote{\foreignlanguage{russian}{Молодой Софроницкий}})
 & Aucun
 & Gravé à la demande \\
 Andante AN~1190
 & UPC~699487119024
 & 2005 \\
 Arbiter ARB~157
 & UPC~604907015725
 & 2008 \\
 Arbiter ARB~164
 & UPC~0604907016425
 & 2017-06-20 \\
 Arioso ARI~700
 & EAN~4560139897002
 & 2003 \\
 Arkadia~78571
 & EAN~8011571785717
 & 2000 \\
 Arlecchino ARL~1 (\Volume{I})
 & EAN~8016811750012
 & 1994* \\
 Arlecchino ARL~2 (\Volume{II})
 & EAN~8016811750029
 & 1994* \\
 Arlecchino ARL~12 (\Volume{III})
 & EAN~8016811750128
 & 1994* \\
 Arlecchino ARL~28 (\Volume{IV})
 & EAN~8016811750289
 & 1995* \\
 Arlecchino ARL~31 (\Volume{V}~; \Scriabine{} \Volume{1})
 & EAN~8016811750319
 & 1995* \\
 Arlecchino ARL~39 (\Volume{VI})
 & EAN~8016811750395
 & 1995* \\
 Arlecchino ARL~40 (\Volume{VII})
 & EAN~8016811750401
 & 1995* \\
 Arlecchino ARL~42 (\Volume{VIII}~; \Scriabine{} \Volume{2})
 & EAN~8016811750425
 & 1995* \\
 Arlecchino ARL~41 (\Volume{IX})
 & EAN~8016811750418
 & 1995* \\
 Arlecchino ARL~61 (\Volume{X})
 & EAN~8016811750616
 & 1995* \\
 Arlecchino ARL~62 (\Volume{XI}~; \Scriabine{} \Volume{3})
 & EAN~8016811750623
 & 1995* \\
 Arlecchino ARL~67 (\Volume{XII})
 & EAN~8016811750678
 & 1995* \\
 Arlecchino ARL~95 (\Volume{XIII})
 & EAN~8016811750951
 & 1995* \\
 Arlecchino ARL~119 (\Volume{XIV}~; \Scriabine{} \Volume{4})
 & EAN~8016811751194
 & 1996* \\
 Arlecchino ARL~132 (\Volume{XV})
 & EAN~8016811751323
 & 1996* \\
 Arlecchino ARL~155 (\Volume{XVI})
 & EAN~8016811751552
 & 1996* \\
 Arlecchino ARL~174 (\Volume{XVII})
 & EAN~8016811751743
 & 1996* \\
 Arlecchino ARL~183 (\Volume{XVIII})
 & EAN~8016811751835
 & 1996* \\
 Arlecchino ARL~188 (\Volume{XIX}~; \Scriabine{} \Volume{5})
 & EAN~8016811751880
 & 1996* \\
 Arlecchino ARL~A11 (\Volume{XX})
 & EAN~8016811752115
 & 1996* \\
 Brilliant Classics BRIL~8975
 & EAN~5029365897525
 & 2008 \\
 Brilliant Classics BRIL~9013
 & EAN~5029365901321
 & 2008 \\
 Brilliant Classics BRIL~9014
 & EAN~5029365901420
 & 2009 \\
 Brilliant Classics BRIL~9241
 & EAN~5029365924122
 & 2011 \\
 Brilliant Classics BRIL~94215
 & EAN~5028421942155
 & 2011 \\
 Classical Recordings Quarterly Editions CD030
 & Inconnu
 & Inconnue \\
 Classical Records CR-004
 & EAN~4603141120048
 & 2002 \\
 Classical Records CR-014
 & EAN~4603141120147
 & 2003 \\
 Classound CLAS~001-022
 & EAN~0889253314634**
 & 2001 \\
 Classound CLAS~001-023
 & EAN~0889253314610**
 & 2001 \\
 Classound CLAS~001-024
 & EAN~0000014000324
 & 2001 \\
 Classound CLAS~001-025
 & EAN~4602002236874**
 & 2001 \\
 Classound CLAS~001-026
 & Inconnu
 & 2001 \\
 Classound CLAS~2003-008
 & EAN~4602002244367**
 & 2003 \\
 Denon COCO-80074/5 (\Volume{1})
 & EAN~4988001302681
 & 1996-03-20 \\
 Denon COCO-80149/50 (\Volume{2})
 & EAN~4988001380887
 & 1996-07-20 \\
 Denon COCO-80187 (\Volume{3})
 & EAN~4988001397281
 & 1996-07-20 \\
 Denon COCO-80188 (\Volume{4})
 & EAN~4988001397380
 & 1996-07-20 \\
 Denon COCO-80189 (\Volume{5})
 & EAN~4988001397489
 & 1996-07-20 \\
 Denon COCO-80383/4 (\Volume{6})
 & EAN~4988001446088
 & 1996-11-21 \\
 Denon COCO-80385/6 (\Volume{7})
 & EAN~4988001446187
 & 1996-11-21 \\
 Denon COCO-80568 (\Volume{8})
 & EAN~4988001296959
 & 1997-05-21 \\
 Denon COCO-80569 (\Volume{9})
 & EAN~4988001295457
 & 1997-05-21 \\
 Denon COCO-80714 (\Volume{10})
 & EAN~4988001071594
 & 1997-11-21 \\
 Denon COCO-80715 (\Volume{11})
 & EAN~4988001071693
 & 1997-11-21 \\
 Denon COCO-80815 (\Volume{12})
 & EAN~4988001138297
 & 1998-04-21 \\
 Denon COCO-80816 (\Volume{13})
 & EAN~4988001138396
 & 1998-04-21 \\
 Denon COCQ-83286 (\Volume{14})
 & EAN~4988001372790
 & 1999-12-18 \\
 Denon COCQ-83490 (\Volume{15})
 & EAN~4988001902508
 & 2000-12-21 \\
 Denon COCQ-83667/8 (\Volume{6}~; réédition \Volume{1})
 & EAN~4988001998129
 & 2003-05-21 \\
 Denon COCQ-83669/70 (\Volume{7}~; réédition \Volume{2})
 & EAN~4988001998228
 & 2003-05-21 \\
 Denon COCQ-83671 (\Volume{8}~; réédition \Volume{3})
 & EAN~4988001998327
 & 2003-05-21 \\
 Denon COCQ-83672 (\Volume{12}~; réédition \Volume{4})
 & EAN~4988001998426
 & 2003-05-21 \\
 Denon COCQ-83673/4 (\Volume{1}~; réédition \Volume{5})
 & EAN~4988001998525
 & 2003-05-21 \\
 Denon COCQ-83968/9 (\Volume{2}~; réédition \Volume{6})
 & EAN~4988001975755
 & 2005-05-25 \\
 Denon COCQ-83970 (\Volume{3}~; réédition \Volume{7})
 & EAN~4988001975854
 & 2005-05-25 \\
 Denon COCQ-83971 (\Volume{14}~; réédition \Volume{8})
 & EAN~4988001975953
 & 2005-05-25 \\
 Denon COCQ-83972 (\Volume{15}~; réédition \Volume{9})
 & EAN~4988001976059
 & 2005-05-25 \\
 Denon COCQ-83973 (\Volume{11}~; réédition \Volume{10})
 & EAN~4988001976158
 & 2005-05-25 \\
 Denon COCQ-84241 (\Volume{9}~; réédition \Volume{11})
 & EAN~4988001993377
 & 2006-11-22 \\
 Denon COCQ-84242 (\Volume{13}~; réédition \Volume{12})
 & EAN~4988001993476
 & 2006-11-22 \\
 Diapason (\Quote{Les indispensables}) DIAP058
 & EAN~3770003441588
 & 2014-04 \\
 DiwClassics DCL-1001
 & EAN~4988044611344
 & 2007-09-21 \\
 DiwClassics DCL-1002
 & EAN~4988044611351
 & 2007-09-21 \\
 Documents~600158 (CD~9)
 & EAN~4053796001580
 & 2014-05-01 \\
 Green Door GDCS-0030
 & EAN~4580139521158
 & 2006 \\
 Harmonia Mundi HMX~1905169
 & EAN~3149025016225**
 & 1986 \\
 Le Chant du monde LDC~278764
 & EAN~3149023387648
 & 1988* \\
 Le Chant du monde LDC~278765
 & EAN~3149023387655
 & 1988* \\
 Le Chant du monde LDC~288032
 & EAN~3149025049896
 & 1992* \\
 Le Monde du piano (\Volume{38})
 & ISBN~\hbox{978-2-36156-040-9}
 & 2010-01 \\
 Marston Records MR~\hbox{54001-2}
 & UPC~638335400129
 & 2010 \\
 Meldac (Triton) MECC~26012
 & EAN~4988030007090
 & 1994-12-16 \\
 Meldac (Triton) MECC~26016
 & EAN~4988030007373
 & 1995-02-22 \\
 Melodija MCD~208
 & EAN~5015524002084
 & 1988 \\
 Melodija MEL CD~10~00747
 & EAN~4600317007479
 & 2004 \\
 Melodija MEL CD~10~02237
 & EAN~4600317122370
 & 2014 \\
 Melodija MEL CD~10~02312
 & EAN~4600317023127
 & 2015 \\
 Melodija MEL CD~10~02395
 & Non commercialisé
 & 2015 \\
 Melodija MEL CD~10~02550
 & EAN~4600317025503
 & 2018-10 \\
 Melodija MEL CD~74321~\hbox{25177-2}
 & UPC~743212517729
 & 1995 \\
 Melodija GP15926 (dématérialisé)
 & EAN~0191773150303
 & 2017-05-15 \\
 Melodija/Bukok DE~0176
 & Inconnu
 & 1997* \\
 Melodija/Bukok DE~0177
 & Inconnu
 & 1997* \\
 Melodija/Bukok DE~0178
 & Inconnu
 & 1997* \\
 Melodija/Bukok DE~0180
 & Inconnu
 & 1997* \\
 Melodija/Bukok DE~0186
 & Inconnu
 & 1997* \\
 Monopole MONO~018
 & EAN~7011778130182
 & 2006 \\
 Moscow State Conservatoire SMC~CD~0019
 & EAN~0889253319455**
 & 1998 \\
 Moscow State Conservatoire SMC~CD~0020
 & EAN~0889254313094**
 & 1998 \\
 Moscow State Conservatoire SMC~CD~0183
 & Inconnu
 & 2016 \\
 Moscow State Conservatoire SMC~CD~0019/0020
 & Inconnu
 & 2017 \\
 Multisonic MT~\hbox{310181-2}
 & UPC~741941018128
 & 1993 \\
 Musikstrasse MC~2108
 & EAN~8004644210810
 & 1994 \\
 Musikstrasse MC~2109
 & EAN~8004644210919
 & 1994 \\
 Olympia OCD~208
 & EAN~5015524002084
 & 1988 \\
 Originals SH~858
 & EAN~8011662904317
 & 1995* \\
 Palladio (Enterprise) PD~4131
 & EAN~8011662900500
 & 1993* \\
 Philips~456~\hbox{970-2}
 & UPC~028945697024
 & 1999 \\
 Profil Hänssler DCD PH15007
 & UPC~881488150070
 & 2015 \\
 Prometheus Editions EDITION003
 & EAN~5027731707133
 & 2002 \\
 Russian Compact Disc RCD~16036
 & EAN~4600383160368
 & 2007 \\
 Russian Compact Disc RCD~16288
 & EAN~4600383162881
 & 1996 \\
 Russian Compact Disc RCD~16289
 & EAN~4600383162898
 & 2011 \\
 Russian Disc RD~CD~15001
 & UPC~748871500129
 & 1993 \\
 Scribendum SC817
 & EAN~5060028048175
 & 2019-11-09 \\
 Selene \hbox{CD-s}~9809.45
 & Inconnu
 & 1998* \\
 The Piano Library PL~282
 & EAN~8011662911995
 & 1998 \\
 TKM Records TNS LS (Living Stage)~4035182
 & EAN~3830025741414
 & 2003 \\
 Urania SP~4203
 & EAN~8025726042037
 & 2002 \\
 Urania SP~4205
 & EAN~8025726042051
 & 2002 \\
 Urania SP~4211
 & EAN~8025726042112
 & 2003 \\
 Urania SP~4215
 & EAN~8025726042150
 & 2003 \\
 Urania SP~4258
 & EAN~8025726042587
 & 2007 \\
 Urania URN~22.299
 & EAN~8025726222996
 & 2006 \\
 Venecija CDVE~00014
 & EAN~2700000000144
 & 2011 \\
 Venecija CDVE~05218
 & Inconnu
 & 2005* \\
 Vista Vera VVCD-00014 (série~1, \Volume{1})
 & EAN~4603141000142
 & 2002 \\
 Vista Vera VVCD-00024 (série~1, \Volume{2})
 & EAN~4603141000241
 & 2002 \\
 Vista Vera VVCD-00031 (série~1, \Volume{3})
 & EAN~4603141000319
 & 2001, 2003 \\
 Vista Vera VVCD-00091 (série~1, \Volume{4})
 & EAN~4603141000913
 & 2005 \\
 Vista Vera VVCD-00093 (série~1, \Volume{5})
 & EAN~4603141000937
 & 2006 \\
 Vista Vera VVCD-00113 (série~1, \Volume{6})
 & EAN~4603141001132
 & 2006 \\
 Vista Vera VVCD-\hbox{00118-2} (série~1, \Volume{7})
 & EAN~4603141001187
 & 2006 \\
 Vista Vera VVCD-00125 (série~1, \Volume{8})
 & EAN~4603141001255
 & 2006 \\
 Vista Vera VVCD-00136 (série~1, \Volume{9})
 & EAN~4603141001361
 & 2007 \\
 Vista Vera VVCD-00137 (série~1, \Volume{10})
 & EAN~4603141001378
 & 2007 \\
 Vista Vera VVCD-\hbox{00148-2} (série~1, \Volume{11})
 & EAN~4603141001484
 & 2007 \\
 Vista Vera VVCD-00155 (série~1, \Volume{12})
 & EAN~4603141001552
 & 2007 \\
 Vista Vera VVCD-00164 (série~1, \Volume{13})
 & EAN~4603141001644
 & 2008 \\
 Vista Vera VVCD-00182 (série~1, \Volume{14})
 & EAN~4603141001828
 & 2008 \\
 Vista Vera VVCD-00198 (série~1, \Volume{15})
 & EAN~4603141001989
 & 2009 \\
 Vista Vera VVCD-00203 (série~1, \Volume{16})
 & EAN~4603141002030
 & 2009 \\
 Vista Vera VVCD-00204 (série~1, \Volume{17})
 & EAN~4603141002047
 & 2009 \\
 Vista Vera VVCD-00218 (série~1, \Volume{18})
 & EAN~4603141002184
 & 2010 \\
 Vista Vera VVCD-00241 (série~1, \Volume{19})
 & EAN~4603141002412
 & 2011 \\
 Vista Vera VVCD-00195 (série~2, \Volume{1})
 & EAN~4603141001958
 & 2009 \\
 Vista Vera VVCD-00210 (série~2, \Volume{2})
 & EAN~4603141002108
 & 2009 \\
 Vista Vera VVCD-00213 (série~2, \Volume{3})
 & EAN~4603141002139
 & 2010 \\
 Vista Vera VVCD-00222 (série~2, \Volume{4})
 & EAN~4603141002221
 & 2010 \\
 Vista Vera VVCD-00223 (série~2, \Volume{5})
 & EAN~4603141002238
 & 2010 \\
 Vista Vera VVCD-00224 (série~2, \Volume{6})
 & EAN~4603141002245
 & 2010 \\
 Vista Vera VVCD-00225 (série~2, \Volume{7})
 & EAN~4603141002252
 & 2010 \\
 Vista Vera VVCD-00233 (série~2, \Volume{8})
 & EAN~4603141002337
 & 2011 \\
 Vista Vera VVCD-00248 (série~2, \Volume{9})
 & EAN~4603141002481
 & 2013 \\
 Vista Vera VVCD-00249 (série~2, \Volume{10})
 & EAN~4603141002498
 & 2013 \\
 Vista Vera VVCD-97014 (série~1, \Volume{1})
 & EAN~0689015001426**
 & 1997
\end{longtable}}
