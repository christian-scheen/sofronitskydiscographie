\chapter[%
Franz Schubert (\Dates{1797-01-31}{1828-11-19})~;
Ferenc Liszt (\Dates{1811-10-22}{1886-07-31})][%
Franz Schubert~; Ferenc Liszt]{%
\FSchubert{} (\Dates{1797-01-31}{1828-11-19})~;\newline
\FLiszt{} (\Dates{1811-10-22}{1886-07-31})}
\label{chap:Schubert-Liszt}

\section{\ifChrono \Schubert{}/\Liszt{}~: \fi
Aufenthalt, D~957 \Number{5} (\Opus{posthume})~; S~560 \Number{3}}
\label{\thesection}
\index[ndxworks]{\indxbf{Schubert/Liszt}!D~957 \Number{5}~; S~560
\Number{3} (Aufenthalt)}

\begin{workitemize}
 \item\Performance{1960-10-11}{3}{25}{\MCSH}{\Live}
 \begin{perfitemize}
  \item\EditionOnLP{%
  Melodija M10 42253/64 (\CR{8})}
  \item\EditionOnCD{%
  \nohand{Arlecchino ARL~2},
  \onhand{Classound CLAS~001-022},
  \nohand{Denon COCO-80383/4},
  \onhand{Denon COCQ-83667/8},
  \onhand{Harmonia Mundi HMX~1905169},
  \nohand{Meldac/Triton MECC~26016},
  \onhand{Scribendum SC817/30},
  \onhand{Vista Vera VVCD-00031}}
 \end{perfitemize}
 \item\Performance{1960-10-14 (incertaine)}{3}{18}{\MCSH}{\Live}
 \Comment{Vista Vera VVCD-00148-2 propose une interprétation différente de
 celle proposée par Harmonia Mundi HMX~1905169, tout en donnant la date
 d'enregistrement 1960-10-11~; la date ci-dessus est incertaine.
 Cependant, l'interprétation de la dernière partie de l'œuvre est semblable,
 de même que les quelques toux dans le public à la fin.
 Scribendum SC817/14 propose la même interprétation que celle proposée par
 Vista Vera VVCD-00148-2.
 Voir tableau~\ref{tab:fsfl}.}
 \begin{perfitemize}
  \item\EditionOnCD{%
  \onhand{Scribendum SC817/14},
  \onhand{Vista Vera VVCD-00148-2}}
 \end{perfitemize}
\end{workitemize}

\section{\ifChrono \Schubert{}/\Liszt{}~: \fi
Der Doppelgänger, D~957 \Number{13} (\Opus{posthume})~; S~560 \Number{12}}
\label{\thesection}
\index[ndxworks]{\indxbf{Schubert/Liszt}!D~957 \Number{13}~; S~560
\Number{12} (Der Doppelgänger)}

\begin{workitemize}
 \item\Performance{1948-01-05}{3}{49}{\MCGH}{\Live}
 \begin{perfitemize}
  \item\EditionOnLP{%
  Melodija D019187/8}
  \item\EditionOnCD{%
  \nohand{Meldac/Triton MECC~26016},
  \onhand{Scribendum SC817/16},
  \onhand{Vista Vera VVCD-00204}}
 \end{perfitemize}
 \item\Performance{1953-12-25}{3}{32}{\MPTH}{\Live}
 \begin{perfitemize}
  \item\EditionOnLP{%
  Melodija M10 42469/78 (\CR{6})}
  \item\EditionOnCD{%
  \onhand{Brilliant Classics BRIL~8975/7},
  \nohand{Brilliant Classics BRIL~94215/28},
  \onhand{Le Monde du piano (\Volume{38}, CD~2)},
  \onhand{Scribendum SC817/04},
  \onhand{Vista Vera VVCD-00113}}
 \end{perfitemize}
 \item\Performance{1955-06-25}{3}{37}{\MSHM}{\Live}
 \begin{perfitemize}
  \item\EditionOnCD{%
  \onhand{Scribendum SC817/24},
  \onhand{Vista Vera VVCD-00210}}
 \end{perfitemize}
 \item\Performance{1960-10-11}{3}{34}{\MCSH}{\Live}
 \Comment{L'interprétation proposée par Vista Vera VVCD-00148-2 est
 identique à celle publiée par Melodija D013539/40, et donc différente de
 celle proposée par Harmonia Mundi HMX~1905169 (1960-10-14).
 Scribendum SC817/14 propose la même interprétation que celle proposée par
 Vista Vera VVCD-00148-2.
 Voir tableau~\ref{tab:fsfl}.}
 \begin{perfitemize}
  \item\EditionOnLP{%
  \onfile{Melodija D013539/40}}
  \item\EditionOnCD{%
  \onhand{Scribendum SC817/14},
  \onhand{Vista Vera VVCD-00148-2}}
 \end{perfitemize}
 \item\Performance{1960-10-14}{3}{26}{\MCSH}{\Live}
 \begin{perfitemize}
  \item\EditionOnLP{%
  Melodija M10 42253/64 (\CR{8})}
  \item\EditionOnCD{%
  \nohand{Arlecchino ARL~2},
  \nohand{Denon COCO-80383/4},
  \onhand{Denon COCQ-83667/8},
  \onhand{Harmonia Mundi HMX~1905169},
  \onhand{Scribendum SC817/30},
  \onhand{Vista Vera VVCD-00031}}
 \end{perfitemize}
\end{workitemize}

\section{\ifChrono \Schubert{}/\Liszt{}~: \fi
Erlkönig, D~328 (\Opus{1})~; S~558 \Number{4}}
\label{\thesection}
\index[ndxworks]{\indxbf{Schubert/Liszt}!D~328 (\Opus{1})~; S~558
\Number{4} (Erlkönig)}

\begin{workitemize}
 \item\Performance{1948-01-05}{4}{52}{\MCGH}{\Live}
 \begin{perfitemize}
  \item\EditionOnCD{%
  \onhand{Scribendum SC817/16},
  \onhand{Vista Vera VVCD-00204}}
 \end{perfitemize}
 \item\Performance{1953-12-25}{4}{56}{\MPTH}{\Live}
 \begin{perfitemize}
  \item\EditionOnLP{%
  Melodija M10 42469/78 (\CR{6})}
  \item\EditionOnCD{%
  \onhand{Brilliant Classics BRIL~8975/7},
  \nohand{Brilliant Classics BRIL~94215/28},
  \onhand{Scribendum SC817/04},
  \onhand{Vista Vera VVCD-00113}}
 \end{perfitemize}
 \item\Performance{1960-10-11}{4}{50}{\MCSH}{\Live}
 \begin{perfitemize}
  \item\EditionOnLP{%
  Melodija M10 42253/64 (\CR{8})}
  \item\EditionOnCD{%
  \nohand{Arlecchino ARL~2},
  \nohand{Denon COCO-80383/4},
  \onhand{Denon COCQ-83667/8},
  \onhand{Harmonia Mundi HMX~1905169},
  \nohand{Meldac/Triton MECC~26016},
  \onhand{Scribendum SC817/30},
  \onhand{Vista Vera VVCD-00031}}
 \end{perfitemize}
 \item\Performance{1960-10-14 (incertaine)}{4}{59}{\MCSH}{\Live}
 \Comment{Vista Vera VVCD-00148-2 propose une interprétation dont certains
 aspects semblent différents de l'interprétation proposée par Harmonia Mundi
 HMX~1905169, tout en donnant la date d'enregistrement 1960-10-11~; la date
 ci-dessus est incertaine.
 Scribendum SC817/14 propose la même interprétation que celle proposée par
 Vista Vera VVCD-00148-2.
 Voir tableau~\ref{tab:fsfl}.}
 \begin{perfitemize}
  \item\EditionOnCD{%
  \onhand{Scribendum SC817/14},
  \onhand{Vista Vera VVCD-00148-2}}
 \end{perfitemize}
\end{workitemize}

\section{\ifChrono \Schubert{}/\Liszt{}~: \fi
Frühlingsglaube, D~686 (\Opus{20} \Number{2})~; S~558 \Number{7}}
\label{\thesection}
\index[ndxworks]{\indxbf{Schubert/Liszt}!D~686 (\Opus{20} \Number{2})~;
S~558 \Number{7} (Frühlingsglaube)}

\begin{workitemize}
 \item\Performance{1953-12-25}{3}{16}{\MPTH}{\Live}
 \begin{perfitemize}
  \item\EditionOnLP{%
  Melodija M10 42469/78 (\CR{6})}
  \item\EditionOnCD{%
  \onhand{Brilliant Classics BRIL~8975/7},
  \nohand{Brilliant Classics BRIL~94215/28},
  \nohand{Meldac/Triton MECC~26016},
  \onhand{Le Monde du piano (\Volume{38}, CD~2)},
  \onhand{Scribendum SC817/04},
  \onhand{Vista Vera VVCD-00113}}
 \end{perfitemize}
 \item\Performance{1960-10-14}{3}{16}{\MCSH}{\Live}
 \begin{perfitemize}
  \item\EditionOnLP{%
  Melodija M10 42253/64 (\CR{8})}
  \item\EditionOnCD{%
  \nohand{Arlecchino ARL~2},
  \nohand{Denon COCO-80383/4},
  \onhand{Denon COCQ-83667/8},
  \onhand{Harmonia Mundi HMX~1905169},
  \onhand{Scribendum SC817/14},
  \onhand{Scribendum SC817/30},
  \onhand{Vista Vera VVCD-00031},
  \onhand{Vista Vera VVCD-00148-2}}
 \end{perfitemize}
\end{workitemize}

\section{\ifChrono \Schubert{}/\Liszt{}~: \fi
Litanei, D~343 (\Opus{posthume})~; S~562 \Number{1}}
\label{\thesection}
\index[ndxworks]{\indxbf{Schubert/Liszt}!D~343~; S~562 \Number{1}
(Litanei)}

\begin{workitemize}
 \item\Performance{1960-10-11}{4}{39}{\MCSH}{\Live}
 \begin{perfitemize}
  \item\EditionOnLP{%
  Melodija M10 42253/64 (\CR{8})}
  \item\EditionOnCD{%
  \nohand{Arlecchino ARL~2},
  \nohand{Arlecchino ARL~183},
  \onhand{Classound CLAS~001-022},
  \nohand{Denon COCO-80383/4},
  \onhand{Denon COCQ-83667/8},
  \nohand{Meldac/Triton MECC~26016},
  \onhand{Scribendum SC817/30},
  \onhand{Vista Vera VVCD-00031}}
 \end{perfitemize}
 \item\Performance{1960-10-14 (incertaine)}{4}{58}{\MCSH}{\Live}
 \Comment{Vista Vera VVCD-00148-2 propose une interprétation différente de
 celle proposée par Denon COCO-80383/4~= COCQ-83667/8, tout en donnant la
 date d'enregistrement 1960-10-11~; la date ci-dessus est incertaine.
 Scribendum SC817/13 propose la même interprétation que celle proposée par
 Vista Vera VVCD-00148-2.
 Voir tableau~\ref{tab:fsfl}.}
 \begin{perfitemize}
  \item\EditionOnCD{%
  \onhand{Scribendum SC817/13},
  \onhand{Vista Vera VVCD-00148-2}}
 \end{perfitemize}
 \item\Performance{1960-12-11}{4}{50}{\Moscow}{\Studio}
 \begin{perfitemize}
  \item\EditionOnLP{%
  \onfile{Melodija M10 43119/30 (\CR{12})}}
  \item\EditionOnCD{%
  \onhand{Melodija MEL CD~10~02237/1},
  \nohand{Melodija MEL CD~10~02395/1}}
 \end{perfitemize}
\end{workitemize}

\section{\ifChrono \Schubert{}/\Liszt{}~: \fi
Am Meer, D~957 \Number{12} (\Opus{posthume})~; S~560 \Number{4}}
\label{\thesection}
\index[ndxworks]{\indxbf{Schubert/Liszt}!D~957 \Number{12}~; S~560
\Number{4} (Am Meer)}

\begin{workitemize}
 \item\Performance{1953-12-25}{3}{58}{\MPTH}{\Live}
 \begin{perfitemize}
  \item\EditionOnLP{%
  Melodija M10 42469/78 (\CR{6})}
  \item\EditionOnCD{%
  \nohand{Arlecchino ARL~2},
  \onhand{Brilliant Classics BRIL~8975/7},
  \nohand{Brilliant Classics BRIL~94215/28},
  \onhand{Le Monde du piano (\Volume{38}, CD~2)},
  \onhand{Scribendum SC817/04},
  \onhand{Vista Vera VVCD-00113}}
 \end{perfitemize}
\end{workitemize}

\section{\ifChrono \Schubert{}/\Liszt{}~: \fi
Der Müller und der Bach, D~795 \Number{19} (\Opus{25} \Number{19})~; S~565
\Number{2}}
\label{\thesection}
\index[ndxworks]{\indxbf{Schubert/Liszt}!D~795 \Number{19} (\Opus{25}
\Number{19})~; S~565 \Number{2} (Der Müller und der Bach)}

\begin{workitemize}
 \item\Performance{1946-08-29}{5}{18}{\Moscow}{\Studio}
 \begin{perfitemize}
  \item\EditionOnLP{%
  Melodija M10 44603/14 (\CR{9})}
  \item\EditionOnCD{%
  \nohand{Denon COCO-80569},
  \onhand{Denon COCQ-84241},
  \nohand{Russian Compact Disc RCD~16288}}
 \end{perfitemize}
 \item\Performance{1948-01-05}{5}{04}{\MCGH}{\Live}
 \begin{perfitemize}
  \item\EditionOnLP{%
  Melodija D019187/8}
  \item\EditionOnCD{%
  \onhand{Scribendum SC817/16},
  \onhand{Vista Vera VVCD-00204}}
 \end{perfitemize}
 \item\Performance{1955-06-25}{5}{09}{\MSHM}{\Live}
 \begin{perfitemize}
  \item\EditionOnCD{%
  \onhand{Scribendum SC817/24},
  \onhand{Vista Vera VVCD-00210}}
 \end{perfitemize}
 \item\Performance{1960-10-11}{4}{52}{\MCSH}{\Live}
 \begin{perfitemize}
  \item\EditionOnLP{%
  Melodija M10 42253/64 (\CR{8})}
  \item\EditionOnCD{%
  \nohand{Arlecchino ARL~2},
  \nohand{Denon COCO-80383/4},
  \onhand{Denon COCQ-83667/8},
  \onhand{Harmonia Mundi HMX~1905169},
  \nohand{Meldac/Triton MECC~26016},
  \onhand{Scribendum SC817/14},
  \onhand{Scribendum SC817/30},
  \onhand{Vista Vera VVCD-00031},
  \onhand{Vista Vera VVCD-00148-2}}
 \end{perfitemize}
\end{workitemize}

\section{\ifChrono \Schubert{}/\Liszt{}~: \fi
Die junge Nonne, D~828 (\Opus{43} \Number{1})~; S~558 \Number{6}}
\label{\thesection}
\index[ndxworks]{\indxbf{Schubert/Liszt}!D~828 (\Opus{43} \Number{1})~;
S~558 \Number{6} (Die junge Nonne)}

\begin{workitemize}
 \item\Performance{1953-12-25}{4}{41}{\MPTH}{\Live}
 \begin{perfitemize}
  \item\EditionOnLP{%
  Melodija M10 42469/78 (\CR{6})}
  \item\EditionOnCD{%
  \nohand{Arlecchino ARL~2},
  \onhand{Brilliant Classics BRIL~8975/7},
  \nohand{Brilliant Classics BRIL~94215/28},
  \onhand{Le Monde du piano (\Volume{38}, CD~2)},
  \onhand{Scribendum SC817/04},
  \onhand{Vista Vera VVCD-00113}}
 \end{perfitemize}
\end{workitemize}

\section{\ifChrono \Schubert{}/\Liszt{}~: \fi
Soirée de Vienne \Number{7}, S~427 \Number{7}, \Quote{Valse-caprice}}
\label{\thesection}
\index[ndxworks]{\indxbf{Schubert/Liszt}!S~427 \Number{7}
(Valse-caprice)}

\begin{workitemize}
 \item\Performance{1957-01-19}{5}{07}{\MSHM}{\Live}
 \begin{perfitemize}
  \item\EditionOnLP{%
  Melodija D019637/8}
  \item\EditionOnCD{%
  \onhand{Scribendum SC817/14},
  \onhand{Vista Vera VVCD-00210}}
 \end{perfitemize}
\end{workitemize}

\section{\ifChrono \Schubert{}/\Liszt{}~: \fi
Die Stadt, D~957 \Number{11} (\Opus{posthume})~; S~560 \Number{1}}
\label{\thesection}
\index[ndxworks]{\indxbf{Schubert/Liszt}!D~957 \Number{11}~; S~560
\Number{1} (Die Stadt)}

\begin{workitemize}
 \item\Performance{1953-12-25}{2}{27}{\MPTH}{\Live}
 \begin{perfitemize}
  \item\EditionOnLP{%
  Melodija M10 42469/78 (\CR{6})}
  \item\EditionOnCD{%
  \nohand{Arlecchino ARL~2},
  \onhand{Brilliant Classics BRIL~8975/7},
  \nohand{Brilliant Classics BRIL~94215/28},
  \onhand{Le Monde du piano (\Volume{38}, CD~2)},
  \onhand{Scribendum SC817/04},
  \onhand{Vista Vera VVCD-00113}}
 \end{perfitemize}
\end{workitemize}

\section{\ifChrono \Schubert{}/\Liszt{}~: \fi
Auf dem Wasser zu singen, D~774 (\Opus{72})~; S~558 \Number{2}}
\label{\thesection}
\index[ndxworks]{\indxbf{Schubert/Liszt}!D~774 (\Opus{72})~; S~558
\Number{2} (Auf dem Wasser zu singen)}

Ce lied est parfois appelé \Quote{Barcarolle} par certains éditeurs.

\begin{workitemize}
 \item\Performance{1948-01-05}{3}{49}{\MCGH}{\Live}
 \begin{perfitemize}
  \item\EditionOnCD{%
  \onhand{Scribendum SC817/16},
  \onhand{Vista Vera VVCD-00204}}
 \end{perfitemize}
 \item\Performance{1953-12-25}{3}{54}{\MPTH}{\Live}
 \begin{perfitemize}
  \item\EditionOnLP{%
  Melodija M10 42469/78 (\CR{6})}
  \item\EditionOnCD{%
  \nohand{Arlecchino ARL~2},
  \onhand{Brilliant Classics BRIL~8975/7},
  \nohand{Brilliant Classics BRIL~94215/28},
  \onhand{Le Monde du piano (\Volume{38}, CD~2)},
  \onhand{Scribendum SC817/04},
  \onhand{Vista Vera VVCD-00113}}
 \end{perfitemize}
 \item\Performance{1955-06-25}{3}{54}{\MSHM}{\Live}
 \begin{perfitemize}
  \item\EditionOnCD{%
  \onhand{Scribendum SC817/24},
  \onhand{Vista Vera VVCD-00210}}
 \end{perfitemize}
\end{workitemize}

\begin{table}[!htbp]
 \centering
 \caption{\Schubert{}/\Liszt{}~: transcriptions de lieder}
 \label{tab:fsfl}
 \begin{tabular}{lll}
  \toprule
  \textbf{Lied} & \textbf{Version \Number{1}} & \textbf{Version \Number{2}}
  \\ \midrule
  \emph{Aufenthalt} & 1960-10-11 & 1960-10-14 (date incertaine) \\
  D~957 \Number{5}~; S~560 \Number{3} & \emph{Melodija \Volume{8}} &
  \emph{Vista Vera VVCD-\hbox{00148-2}} \\
  & & \hspace*{3mm}\dvssymbol{"2937}~en partie distincte de~1 \\
  & & \hspace*{3mm}\dvssymbol{"2937}~date incertaine \\ \midrule
  \emph{Der Doppelgänger} & 1960-10-11 & 1960-10-14 \\
  D~957 \Number{13}~; S~560 \Number{12} & \emph{Melodija D013539/40} &
  \emph{Melodija \Volume{8}} \\
  & Vista Vera VVCD-\hbox{00148-2} & \\
  & \hspace*{3mm}\dvssymbol{"2937}~distincte de~2 & \\ \midrule
  \emph{Erlkönig} & 1960-10-11 & 1960-10-14 (date incertaine) \\
  D~328 (\Opus{1})~; S~558 \Number{4} & \emph{Melodija \Volume{8}} &
  \emph{Vista Vera VVCD-\hbox{00148-2}} \\
  & & \hspace*{3mm}\dvssymbol{"2937}~peut-être distincte de~1 \\
  & & \hspace*{3mm}\dvssymbol{"2937}~date incertaine \\ \midrule
  \emph{Litanei} & 1960-10-11 & 1960-10-14 (date incertaine) \\
  D~343~; S~562 \Number{1} & \emph{Melodija \Volume{8}} & \emph{Vista Vera
  VVCD-\hbox{00148-2}} \\
  & & \hspace*{3mm}\dvssymbol{"2937}~distincte de~1 \\
  & & \hspace*{3mm}\dvssymbol{"2937}~date incertaine \\ \bottomrule
 \end{tabular}
\end{table}

\addtocounter{totperfs}{\value{perfbyauthor}}
\addtocounter{totworks}{\value{section}}
\write\Outline{\FSchubert/\FLiszt~: \arabic{section}~œuvre(s) jouée(s),
\theperfbyauthor~interprétation(s).}
