\chapter[%
Sessions d'enregistrements en studio][%
Sessions d'enregistrements en studio]{%
Sessions d'enregistrements en studio}
\label{chap:Studio}

On connaît la méfiance de \VSofronitsky{} envers le travail réalisé dans les
studios d'enregistrement.
Son verdict habituel, concernant cette partie de son activité de musicien,
était sans appel et est assez souvent documenté~: «~Ne me parlez pas de mes
disques, ce sont tous des cadavres, il faut les détruire~!~»
(\citealt{Nikonovich08b}.)
Du reste, plusieurs bandes d'enregistrements en studio ont en effet été
détruites à sa demande~: «~Démagnétisez~!~»
C'est à l'intervention de ses enfants que l'on doit la préservation de
plusieurs de ses enregistrements en studio \citep[voir][p.~62]{Juban}.
Des propos de la fille aînée du pianiste, \RKoganSofronitskaya{}, rapportés
par \citet[p.~98]{Artese} dans sa thèse de doctorat, évoquent cette méfiance
-- mais l'enregistrement des Études symphoniques de \Schumann{} est issu
d'un récital en public et non d'un travail en studio.

\begin{quotation}
 \textsc{Question}~(A.~A.).--
 \foreignlanguage{english}{\emph{%
  What did he think about his recordings?}}

 «~Que pensait-il de ses enregistrements~?~»

 \textsc{Réponse}~(R.~K.-S.).--
 \foreignlanguage{english}{\emph{%
  He hated all his recordings.
  He only liked his live concerts.
  The recordings, he thought that they were too dry, uninspired, but that's
  [...] what he said during his last years... the earlier times, I~know from
  some of the letters to my mother, I~know that some of the recordings he
  liked... ``Vers la flamme...'' and \Schumann{}... he liked the Symphonic
  Etudes...}}

 «~Il détestait tous ses enregistrements.
 Il n'appréciait que ses concerts en direct.
 Les enregistrements, il pensait qu'ils étaient trop secs, dépourvus
 d'inspiration, mais c'est ce qu'il disait pendant ses dernières années...
 plus tôt, je sais grâce à quelques-unes des lettres à ma mère, je sais
 qu'il appréciait certains des enregistrements... «~Vers la flamme...~» et
 \Schumann{}... il appréciait les Études symphoniques...~»%
\sourceatright{\citep{Artese}}
\end{quotation}

Cependant, sa relation avec le processus d'enregistrement n'était pas si
univoque que cela~; en réalité, elle était assez contradictoire.
Ainsi, à la fin de son entretien avec \citet{Vitsinsky}, qui date du~28
octobre~1945, \VSofronitsky{} évoque une de ses sessions d'enregistrements
en studio dans des termes apaisés sinon élogieux.

\begin{quotation}
 \foreignlanguage{english}{\emph{%
  Yesterday I~performed for a recording.
  You know, it is very useful to listen to yourself on record.
  It gives a performer a great deal.}}

 «~Hier, j'ai joué pour un enregistrement.
 Vous savez, il est très utile d'écouter ses propres enregistrements.
 Cela apporte beaucoup à un musicien.~»%
\sourceatright{\citep{Vitsinsky}}
\end{quotation}

De même, \citet{Nikonovich08b} indique que \VSofronitsky{} était attiré par
le processus d'enregistrement~; l'enregistrement exigeait, selon lui, la
simplicité, et l'arbitraire y était inacceptable -- voir
\citet[p.~567]{Voskobojnikov09b} pour la traduction en italien.
\citet{Nikonovich08b} ajoute que les enregistrements des années~1958 à~1960
étaient excellents et que \VSofronitsky{} lui-même en reconnaissait la bonne
réussite -- voir \citet[p.~567]{Voskobojnikov09b}.
Dans ses notes pour l'édition sur disque vinyle des enregistrements complets
du pianiste, \citet{Nikonovich79} mentionne que \VSofronitsky{} préférait
lui-même ses enregistrements du Poème de \Scriabine{} \emph{Vers la flamme}
(janvier~1959)~; du \emph{Sposalizio} de \Liszt{} (décembre~1960)~; du lied
\emph{Litanei} de \Schubert{} transcrit par \Liszt{} (décembre~1960)~; du
Troisième Impromptu de \Schubert{} (août~1960)~; des Mazurkas, des Nocturnes
et de la Barcarolle de \Chopin{} (janvier et février~1960)~; des Études, des
Poèmes \Opus{32}, des Danses \Opus{73} et de la Fantaisie \Opus{28} de
\Scriabine{} (septembre~1959, octobre~1958, janvier~1959 et avril~1959,
respectivement) -- voir \citet[p.~20]{White}.
On trouve une ambivalence comparable en ce qui concerne l'enregistrement de
ses récitals en public.
Avant~1959, on ne dispose que d'un seul enregistrement de récital par an.
À partir de la saison~1959-1960, les récitals enregistrés sont beaucoup plus
nombreux, et \VSofronitsky{} aurait souhaité qu'ils soient tous enregistrés
afin de pouvoir opérer lui-même une sélection parmi ces bandes en vue de la
publication de nouveaux disques -- voir \citet{Nikonovich79}, et
\citet[p.~20]{White} pour la traduction en anglais.

À propos des enregistrements en studio de \VSofronitsky{} durant ses deux
dernières années, en~1959 et~1960, \citet[p.~403-408]{Shiryaeva} mentionne
que le musicien n'appréciait pas l'instrument du studio et qu'il préférait
la petite salle du conservatoire de Moskva et son instrument, qui n'étaient
cependant disponibles que de nuit, après~23\up{h}, dès la fin des concerts
qui y étaient organisés.
\citeauthor{Shiryaeva} évoque en particulier les enregistrements~:
\begin{itemize}
 \item
 de Moments musicaux de \FSchubert{}, D~780 \Number{1} à \Number{4} et
 \Number{6} (1959-02-27, p.~403-404), un projet préparé dès l'automne~1958~;
 \item
 de la Fantaisie d'\AScriabine{} en \kB mineur, \Opus{28} (1959-04-29,
 p.~404), à une époque où \VSofronitsky{} n'était pas encore rétabli de ses
 graves problèmes cardiaques au printemps~1959~;
 \item
 de la Sonate-fantaisie \Number{2} d'\AScriabine{} en \kG \Sharp mineur,
 \Opus{19} (1960-12-11, p.~404-405), un projet préparé dès l'été~1959 mais
 remplacé par des Études de \Scriabine{}, \Opus{8} \Number{2, 4, 5, 8, 9
 et~11} (1959-09-04 et 1959-09-12, p.~405)~;
 \item
 de dix-huit œuvres de \FChopin{} (1960-01-08, 1960-01-28, 1960-02-11
 et~1960-02-19, p.~405-406), un projet resté incomplet en vue du jubilé
 du~150\ieme{} anniversaire de la naissance du compositeur.
\end{itemize}

\citet[p.~407-408]{Shiryaeva} évoque en outre les circonstances des
enregistrements en public au musée \Scriabine{}, à l'initiative et sous la
supervision de Vadim Krjukov,\index[ndxnames]{Krjukov, Vadim Konstantinovič}
ainsi que quelques disques publiés à partir des récitals du~8 janvier, du~2
février, du~13 mai, des~11 et~14 octobre et des~22 et~28 octobre~1960.

Les sections suivantes recensent, de manière chronologique, les sessions
d'enregistrements en studio réalisées par \VSofronitsky{}.
On y remarque en particulier la raréfaction des enregistrements en studio
entre~1954 et~1957.

\section{Année~1937}

\begin{description}
 \item[1937 à~1941 (date imprécise)]
 \Chopin{} (Mazurka, \Opus{41} \Number{2}).
 \item[\DateWithWeekDay{1937-06-16}]
 \Chopin{} (Étude, \Opus{10} \Number{4}~; Mazurka, \Opus{41} \Number{2}).
 Jusqu'ici, la Mazurka n'a jamais été publiée que sur le disque~78~tours
 numéro~5655/6 en~URSS.
 Ce disque~78~tours n'est mentionné que dans la discographie établie par
 \citet[p.~1]{Nikonovich11}.
 Il est repris dans celle due à \citet[p.~65]{Malik}.
 Voir aussi \citet[p.~375]{Scriabine}.
 Denon COCO-80188 donne la date de cet enregistrement mais il s'agit de
 l'enregistrement du~17 novembre~1938 \citep[voir][p.~65 et note~3]{Malik}.
 \item[\DateWithWeekDay{1937-06-25}]
 \Liszt{} (Gnomenreigen), \Scriabine{} (Étude, \Opus{8} \Number{10}).
\end{description}

\section{Année~1938}

\begin{description}
 \item[\DateWithWeekDay{1938-03-28}]
 \Scriabine{} (Poème, \Opus{32} \Number{1}~; Deux Préludes, \Opus{27}).
 \item[\DateWithWeekDay{1938-11-17}]
 \Chopin{} (Mazurkas, \Opus{41} \Number{2} et \Opus{63} \Number{2}),
 \Goltz{} (Prélude \Number{4}~; Scherzo).
\end{description}

\section{Année~1939}

\begin{description}
 \item[1939 (date imprécise)]
 \Scriabine{} (Prélude, \Opus{27} \Number{1}).
\end{description}

\section{Année~1941}

\begin{description}
 \item[1941-05 (date imprécise)]
 \Chopin{} (Valses, \Opus{64} \Number{1} et \Opus{70} \Number{1}).
 \item[1941-06 (date imprécise)]
 \Schumann{} (Bunte Blätter, \Opus{99} \Number{1}, \Number{6} et
 \Number{7}).
\end{description}

\section{Année~1945}

\begin{description}
 \item[\DateWithWeekDay{1945-01-16}]
 \Rachmaninov{} (Prélude, \Opus{32} \Number{5}), \Scriabine{} (Étude,
 \Opus{8} \Number{11}).
 \item[\DateWithWeekDay{1945-01-18}]
 \Rachmaninov{} (Prélude, \Opus{32} \Number{12}).
 \item[\DateWithWeekDay{1945-01-30}]
 \Scriabine{} (Étude, \Opus{8} \Number{9}).
 \item[\DateWithWeekDay{1945-05-22}]
 \Chopin{} (Valses, \Opus{64} \Number{3} et \Opus{69} \Number{1}).
 \item[\DateWithWeekDay{1945-08-28}]
 \Scriabine{} (Études, \Opus{8} \Number{10} et \Number{12}~; Préludes,
 \Opus{11} \Number{14}, \Opus{13} \Number{6}, \Opus{17} \Number{4} et
 \Opus{31} \Number{3}).
\end{description}

\section{Année~1946}

\begin{description}
 \item[1946 (date imprécise)]
 \Chopin{} (Mazurkas, \Opus{33} \Number{3} et \Opus{41} \Number{2}~;
 Polonaise, \Opus{26} \Number{1}~; Préludes, \Opus{28} \Number{2} et
 \Number{13}~; Valse, \Opus{70} \Number{1}), \Prokofiev{} (Sarcasme,
 \Opus{17} \Number{3}~; Vision fugitive, \Opus{22} \Number{7}),
 \Rachmaninov{} (Études-tableaux, \Opus{33} \Number{7} et \Opus{39}
 \Number{6}~; Moment musical, \Opus{16} \Number{5}~; Préludes, \Opus{23}
 \Number{1}, \Number{4} et \Number{6}~; Préludes, \Opus{32} \Number{3},
 \Number{5} et \Number{12}), \Scriabine{} (Étude, \Opus{8} \Number{9}~;
 Morceaux, \Opus{52} \Number{2}, \Opus{56} \Number{2} et \Opus{57}
 \Number{1}~; Poèmes, \Opus{32} \Number{2}, \Opus{36} et \Opus{44}
 \Number{1} et \Number{2}~; Préludes, \Opus{13} \Opus{6}, \Opus{22}
 \Number{1}, \Opus{27} \Number{1} et \Opus{31} \Number{3}~; Sonate,
 \Opus{23}~; Valse, \Opus{38}).
 \item[1946 (date incertaine)]
 \Scriabine{} (Études, \Opus{8} \Number{4} et \Number{8}).
 La mention de date incertaine provient de \citet[vol.~10]{Nikonovich79}.
 \item[\DateWithWeekDay{1946-08-27}]
 \Chopin{} (Valses, \Opus{64} \Number{3} et \Opus{69} \Number{1}).
 \item[\DateWithWeekDay{1946-08-29}]
 \Schubert{}/\Liszt{} (Der Müller und der Bach).
 \item[\DateWithWeekDay{1946-12-02}]
 \Prokofiev{} (Quatre Contes de la vieille grand-mère, \Opus{31}).
\end{description}

\section{Année~1947}

\begin{description}
 \item[1947 (date imprécise)]
 \Chopin{} (Mazurkas, \Opus{30} \Number{3} et \Number{4}~; Polonaise,
 \Opus{53}~; Valse, \Opus{64} \Number{1}).
 \item[\DateWithWeekDay{1947-07-02}]
 \Medtner{} (Conte, \Opus{20} \Number{2}), \Prokofiev{} (Conte de la vieille
 grand-mère, \Opus{31} \Number{3}).
 \item[\DateWithWeekDay{1947-07-09}]
 \Rachmaninov{} (Moment musical, \Opus{16} \Number{3}~; Prélude, \Opus{23}
 \Number{4}).
 \item[\DateWithWeekDay{1947-08-27}]
 \Chopin{} (Mazurkas, \Opus{50} \Number{3} et \Opus{68} \Number{2}).
 \item[\DateWithWeekDay{1947-09-13}]
 \Chopin{} (Mazurka, \Opus{50} \Number{3}).
 \item[1947-09-22 (date incertaine)]
 \Rachmaninov{} (Étude-tableau, \Opus{39} \Number{6}).
 La mention de date incertaine provient de \citet[vol.~10]{Nikonovich79}.
\end{description}

\section{Année~1948}

\begin{description}
 \item[1948 (date imprécise)]
 \Scriabine{} (Étude, \Opus{42} \Number{4}~; Impromptu, \Opus{14}
 \Number{2}~; Préludes, \Opus{9} \Number{1}, \Opus{13} \Number{1}, \Opus{17}
 \Number{3}, \Opus{35} \Number{2}, \Opus{37} \Number{1} et \Opus{39}
 \Number{4}).
 \item[1948 à~1953 (date imprécise)]
 \Scriabine{} (Prélude, \Opus{35} \Number{2}).
 \item[\DateWithWeekDay{1948-01-13}]
 \Chopin{} (Polonaise, \Opus{26} \Number{1}).
 \item[\DateWithWeekDay{1948-01-15}]
 \Schumann{} (Sonate, \Opus{11} -- deux premiers mouvements).
 \item[\DateWithWeekDay{1948-04-14}]
 \Scriabine{} (Valse, \Opus{38}).
 \item[\DateWithWeekDay{1948-07-05}]
 \Scriabine{} (Prélude, \Opus{16} \Number{1}).
 \item[\DateWithWeekDay{1948-07-06}]
 \Scriabine{} (Étude, \Opus{8} \Number{7}~; Mazurkas, \Opus{25} \Number{3}
 et \Opus{40} \Number{2}).
 \item[\DateWithWeekDay{1948-10-22}]
 \Chopin{} (Étude, \Opus{10} \Number{6}~; Nouvelle Étude \Number{2} en \kA
 \Flat majeur (1839)~; Nocturne, \Opus{15} \Number{1}).
\end{description}

\section{Année~1949}

\begin{description}
 \item[1949 (date imprécise)]
 \Scriabine{} (Morceau, \Opus{2} \Number{1}).
 \item[\DateWithWeekDay{1949-03-09}]
 \Liszt{} (Sposalizio, Il penseroso, Canzonetta del Salvator Rosa).
 \item[\DateWithWeekDay{1949-04-14}]
 \Liadov{} (Morceaux, \Opus{31} \Number{2} et \Opus{57} \Number{1}~;
 Préludes, \Opus{36} \Number{3} et \Opus{40} \Number{3}).
 \item[\DateWithWeekDay{1949-06-14}]
 \Chopin{} (Prélude, \Opus{28} \Number{15}), \Liadov{} (Morceau, \Opus{11}
 \Number{1}~; Une Tabatière à musique, \Opus{32}).
 \item[\DateWithWeekDay{1949-06-19}]
 \Rachmaninov{} (Morceau de fantaisie, \Opus{3} \Number{2}).
 \item[\DateWithWeekDay{1949-07-15}]
 \Liadov{} (Morceau, \Opus{31} \Number{2}~; Trois Morceaux, \Opus{57}~;
 Novelette, \Opus{20}).
\end{description}

\section{Année~1950}

\begin{description}
 \item[1950 (date imprécise)]
 \Chopin{} (Préludes, \Opus{28} \Number{4}, \Number{9} et \Number{10}~;
 Prélude, \Opus{45}), \Mendelssohn{} (Variations sérieuses, \Opus{54}),
 \Scriabine{} (Études, \Opus{8} \Number{1}, \Number{2} et \Number{5}~;
 Études, \Opus{42} \Number{2} et \Number{6}~; Impromptu, \Opus{12}
 \Number{2}).
 \item[1950 à~1960 (date imprécise)]
 \Scriabine{} (Prélude, \Opus{11} \Number{22}).
 \item[\DateWithWeekDay{1950-07-01}]
 \Chopin{} (Nocturne, \Opus{9} \Number{2}).
 \item[\DateWithWeekDay{1950-07-06}]
 \Chopin{} (Prélude, \Opus{28} \Number{21}~; Valse \Opus{70} \Number{3}).
 \item[\DateWithWeekDay{1950-07-12}]
 \Borodine{} (Petite Suite).
 \item[\DateWithWeekDay{1950-07-21}]
 \Chopin{} (Mazurka, \Opus{41} \Number{1}~; Valse, \Opus{70} \Number{2}).
 \item[\DateWithWeekDay{1950-08-01}]
 \Chopin{} (Mazurka, \Opus{68} \Number{3}).
 \item[\DateWithWeekDay{1950-11-03}]
 \Scriabine{} (Polonaise, \Opus{21}).
 \item[\DateWithWeekDay{1950-12-23}]
 \Chopin{} (Prélude, \Opus{28} \Number{2}).
 \item[\DateWithWeekDay{1950-12-27}]
 \Chopin{} (Préludes, \Opus{28} \Number{1}, \Number{13}, \Number{14} et
 \Number{15}).
\end{description}

\section{Année~1951}

\begin{description}
 \item[1951 (date imprécise)]
 \Rachmaninov{} (Études-tableaux, \Opus{33} \Number{2}  et \Opus{39}
 \Number{5}), \Schumann{} (Carnaval, \Opus{9}), \Scriabine{} (Préludes,
 \Opus{11} \Number{1} à \Number{13}, \Number{15} à \Number{17} et
 \Number{19} à \Number{22}~; Préludes, \Opus{31} \Number{1}, \Opus{39}
 \Number{2} et \Opus{48} \Number{2}).
 \item[\DateWithWeekDay{1951-02-07}]
 \Chopin{} (Préludes, \Opus{28} \Number{6}, \Number{7} et \Number{8}).
 \item[\DateWithWeekDay{1951-02-08}]
 \Chopin{} (Préludes, \Opus{28} \Number{17} et \Number{22}).
 \item[\DateWithWeekDay{1951-04-23}]
 \Liadov{} (Novelette, \Opus{20}).
 \item[\DateWithWeekDay{1951-05-05}]
 \Chopin{} (Préludes, \Opus{28} \Number{3}, \Number{5} et \Number{12}).
 \item[\DateWithWeekDay{1951-05-14}]
 \Chopin{} (Préludes, \Opus{28} \Number{18}, \Number{19} et \Number{23}).
 \item[\DateWithWeekDay{1951-08-03}]
 \Chopin{} (Ballade, \Opus{47}).
 \item[\DateWithWeekDay{1951-09-01}]
 \Beethoven{} (Sonate, \Opus{111}).
 \item[\DateWithWeekDay{1951-12-27}]
 \Chopin{} (Préludes, \Opus{28} \Number{11} et \Number{20}).
\end{description}

\section{Année~1952}

\begin{description}
 \item[1952 (date imprécise)]
 \Glazounov{} (Prélude et fugue, \Opus{101} \Number{1}), \Rachmaninov{}
 (Moment musical, \Opus{16} \Number{3}), \Scriabine{} (Préludes, \Opus{16}
 \Number{3} et \Number{4}, \Opus{17} \Number{1}, \Number{4} et \Number{6}).
 \item[\DateWithWeekDay{1952-07-01}]
 \Chopin{} (Nocturne, \Opus{27} \Number{2}), \Glazounov{} (Morceau,
 \Opus{49} \Number{1}), \Scriabine{} (Mazurkas, \Opus{3} \Number{6} et
 \Number{9}).
 \item[\DateWithWeekDay{1952-07-07}]
 \Schumann{} (Kreisleriana, \Opus{16}).
 \item[\DateWithWeekDay{1952-07-14}]
 \Chopin{} (Scherzo, \Opus{20}).
 \item[\DateWithWeekDay{1952-07-22}]
 \Schumann{} (Papillons, \Opus{2}).
 \item[\DateWithWeekDay{1952-07-23}]
 \Liszt{} (Églogue, Sonetto~123 del Petrarca), \Schumann{} (Arabesque,
 \Opus{18}), \Scriabine{} (Mazurka, \Opus{25} \Number{8}).
 \item[\DateWithWeekDay{1952-11-24}]
 \Mozart{} (Fantaisie, K~475).
 \item[\DateWithWeekDay{1952-11-28}]
 \Beethoven{} (Andante favori, WoO~57), \Mozart{} (Fantaisie, K~396).
\end{description}

\section{Année~1953}

\begin{description}
 \item[1953 (date imprécise)]
 \Kabalevski{} (Sonatine, \Opus{13} \Number{1}), \Schubert{} (Impromptu,
 D~899 \Number{1}), \Scriabine{} (Morceaux, \Opus{2} \Number{2} et
 \Number{3}~; Poème, \Opus{44} \Number{2}~; Préludes, \Opus{31} \Number{2}
 et \Number{4}, \Opus{33} \Number{1}, \Number{2} et \Number{3}, \Opus{39}
 \Number{3} et \Number{4}).
 \item[\DateWithWeekDay{1953-01-23}]
 \Scriabine{} (Mazurka, \Opus{25} \Number{7}).
 \item[\DateWithWeekDay{1953-02-06}]
 \Schumann{} (Sonate, \Opus{14} -- troisième mouvement).
 \item[\DateWithWeekDay{1953-02-10}]
 \Schumann{} (Novelette, \Opus{21} \Number{8}).
 \item[\DateWithWeekDay{1953-02-26}]
 \Beethoven{} (Sonate, \Opus{28}).
 \item[\DateWithWeekDay{1953-04-11}]
 \Rachmaninov{} (Étude-tableau, \Opus{39} \Number{4}).
 \item[\DateWithWeekDay{1953-04-13}]
 \Schumann{} (Novelette, \Opus{21} \Number{1}).
 \item[\DateWithWeekDay{1953-05-05}]
 \Chopin{} (Prélude, \Opus{28} \Number{1}).
 \item[\DateWithWeekDay{1953-06-09}]
 \Prokofiev{} (Pièces, \Opus{12} \Number{2}, \Number{3} et \Number{6}).
 \item[\DateWithWeekDay{1953-06-11}]
 \Prokofiev{} (Pièce, \Opus{12} \Number{7}).
 \item[\DateWithWeekDay{1953-06-18}]
 \Prokofiev{} (Pièces, \Opus{12} \Number{8} et \Number{9}).
 \item[\DateWithWeekDay{1953-07-06}]
 \Chopin{} (Étude, \Opus{10} \Number{3}~; Mazurka, \Opus{17} \Number{3}),
 \Schumann{} (Bunte Blätter, \Opus{99} \Number{1} à \Number{8}).
 \item[\DateWithWeekDay{1953-07-11}]
 \Chopin{} (Mazurka, \Opus{24} \Number{1}).
\end{description}

\section{Année~1956}

\begin{description}
 \item[\DateWithWeekDay{1956-01-25}]
 \Schubert{} (Sonate, D~960).
\end{description}

\section{Année~1957}

\begin{description}
 \item[\DateWithWeekDay{1957-11-08}]
 \Prokofiev{} (Pièce, \Opus{12} \Number{1}).
\end{description}

\section{Année~1958}

\begin{description}
 \item[\DateWithWeekDay{1958-09-12}]
 \Scriabine{} (Sonates, \Opus{23} et \Opus{68}).
 \item[\DateWithWeekDay{1958-10-31}]
 \Scriabine{} (Deux Poèmes, \Opus{32}~; Préludes, \Opus{11} \Number{16},
 \Opus{35} \Number{2} et \Opus{37} \Number{1}).
\end{description}

\section{Année~1959}

\begin{description}
 \item[\DateWithWeekDay{1959-01-06}]
 \Scriabine{} (Deux Danses, \Opus{73}~; Poème, \Opus{72}).
 \item[\DateWithWeekDay{1959-02-27}]
 \Schubert{} (Moments musicaux, D~780 \Number{1} à \Number{4} et
 \Number{6}).
 \item[\DateWithWeekDay{1959-04-29}]
 \Scriabine{} (Fantaisie, \Opus{28}).
 \item[\DateWithWeekDay{1959-09-04}]
 \Scriabine{} (Études, \Opus{8} \Number{2}, \Number{4}, \Number{8},
 \Number{9} et \Number{11}).
 \item[\DateWithWeekDay{1959-09-12}]
 \Scriabine{} (Études, \Opus{8} \Number{5} et \Opus{42} \Number{3}).
\end{description}

\section{Année~1960}

\begin{description}
 \item[\DateWithWeekDay{1960-01-08}]
 \Chopin{} (Polonaise, \Opus{26} \Number{1}).
 Enregistrement effectué le même jour qu'un récital \Scriabine{}.
 \item[\DateWithWeekDay{1960-01-28}]
 \Chopin{} (Barcarolle, \Opus{60}~; Impromptu, \Opus{51}~; Deux Nocturnes,
 \Opus{27}).
 \item[\DateWithWeekDay{1960-02-11}]
 \Chopin{} (Mazurkas, \Opus{30} \Number{2} à \Number{4}, \Opus{33}
 \Number{3} et \Number{4}, \Opus{41} \Number{1} et \Number{2}, \Opus{50}
 \Number{3}, \Opus{63} \Number{2} et \Opus{68} \Number{4}).
 \item[\DateWithWeekDay{1960-02-19}]
 \Chopin{} (Valses, \Opus{69} \Number{1} et \Opus{70} \Number{2} et
 \Number{3}).
 \item[\DateWithWeekDay{1960-08-19}]
 \Schubert{} (Impromptus, D~899 \Number{3} et D~935 \Number{2}).
 \item[\DateWithWeekDay{1960-12-11}]
 \Liszt{} (Sposalizio), \Schubert{}/\Liszt{} (Litanei), \Scriabine{}
 (Sonate-fantaisie, \Opus{19} -- premier mouvement).
\end{description}
